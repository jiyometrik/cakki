\chapter{Clauses}
\label{ch:clauses}
This chapter discusses the basic clause structures of \langname{}.

\section{Syntax}
\langname{} features \detail{symmetrical voice}, where the role of the primary argument
(or the argument \detail{in focus}) is encoded onto the verb itself. There are four voices in
\langname{}: \detail{actor voice} (\AV), \detail{patient voice} (\PV), \detail{locative voice} (\LV)
and \detail{instrumental voice} (\IV).
\begin{examples}
	\ex
	\label{ex:actor_voice}
	\script ngain yeo aw amospe \\
	\bits nga -in yeo aw amospe \\
	\gloss eat AV child OBL crab \\
	\tr A child eats crab.
\end{examples}

\begin{examples}
	\ex
	\label{ex:patient_voice}
	\script nganak amospe ya yeo \\
	\bits nga -nak amospe ya yeo \\
	\gloss eat PV crab GEN child \\
	\tr Crab is eaten by the child.
\end{examples}

\begin{examples}
	\ex
	\label{ex:locative_voice}
	\script ngaong sekio aw amospe ya yeo \\
	\bits nga -ong sekio aw amospe ya yeo \\
	\gloss eat LV hut OBL crab GEN child \\
	\tr A hut is eaten crab in by a child.
\end{examples}

\begin{examples}
	\ex
	\label{ex:instrumental_voice}
	\script ngapaw yeo roakyo amospe \\
	\bits nga -paw yeo roa =kyo aw amospe \\
	\gloss eat IV child hand 3SG.POSS OBL crab \\
	\tr A hut is eaten crab in by a child.
\end{examples}

For transitive clauses, \langname{} adopts a verb-initial word order.
This is true for monovalent (or \detail{intransitive}) verbs too, where \AV~is the default.
\begin{examples}
	\ex
	\label{ex:the_child_dies}
	\script rohakin yeo
	\bits rohak =in yeo
	\gloss die AV child
	\tr A child dies.
\end{examples}

Verbs that take no arguments (i.e. have \detail{zero valency}) are left alone, without a voice.
\begin{examples}
	\ex
	\label{ex:its_raining}
	\script rea poipoi
	\bits rea poi^ poi
	\gloss now PROG rain
	\tr It's raining now.
\end{examples}

\section{Nominal clauses}
\langname~lacks a copula. This is easily rectified by a verbalizer conveying the meaning of `to be'---\rom{ko-} (from \rom{koay} `to live').
\begin{examples}
	\ex
	\label{ex:you_bitch}
	\script koyebao
	\bits ko- yeba =ao
	\gloss VBZ.be dog 2SG.NOM
	\tr You're a dog.
\end{examples}

Similarly, \langname~lacks a distinct verb for `to have'---\rom{ma-} (from \rom{maya} `to carry').
\begin{examples}
	\ex
	\label{ex:you_have_a_dog}
	\script mayebao
	\bits ma- yeba =ao
	\gloss VBZ.have dog 2SG.NOM
	\tr You have a dog.
\end{examples}

\section{Dependent clauses}
Dependent clauses are concatenated onto the main clause in most circumstances.
\begin{examples}
	\ex
	\label{ex:boy_want_mom_cook}
	\script sirin hea aw kohanin amha
	\bits sir -in hea aw kohan -in amha
	\gloss want AV boy OBL cook AV mother
	\tr A boy wants his mother to cook.
\end{examples}
\begin{examples}
	\ex
	\label{ex:he_started_beating_the_pig}
	\script haskiin kew aw dasohin bao
	\bits haski -in kew aw dasoh -in bao
	\gloss start AV 3SG.NOM OBL beat AV pig
	\tr He starts to beat the pig.
\end{examples}
The clunkier way to clarify this is to \detail{nominalize} the dependent clause,
with \INST~serving as a converb. No voice is assigned to dependent monotransitive verbs.
\begin{examples}
	\ex
	\label{ex:go_forest_to_see_river}
	\script tsipasoong bohon pa hori-aresokyo
	\bits tsi= paso -ong bohon pa hori are- so =kyo
	\gloss 1SG.NOM walk LV forest INST river NMZ see 3SG.POSS
	\tr A forest is walked in by me with (intentions of) a seeing of a river.
	\tr I walk in a forest to see a river.
\end{examples}

\section{Relative clauses}
The actor in a relative clause is often the same as in its surrounding independent clause.
\LOC~is used to mark the actor's state of being for \detail{stative verbs}.
\begin{examples}
	\ex
	\label{ex:father_who_cried}
	\script ando ong aresabow
	\bits ando ong are- sabow
	\gloss father LOC NMZ cry
	\tr a father in (the act of) crying
	\tr a crying father
\end{examples}

The combo of \INST~and \NMZ~also works for relative clauses with ditransitive verbs.
but the roles of the new nouns introduced must be marked too.
\begin{examples}
	\ex
	\label{ex:cat_who_bit_girl}
	\script podendain miw pa boyan-areingingkakyo aw tsoetsoe \\
	\bits po- denda -in miw pa boyan are- ingik -nak =kyo aw tsoe^ tsoe \\
	\gloss PFV catch AV cat INST girl NMZ bite PV 3SG.POSS OBL COL fish \\
	\tr A cat caught fish with a girl's (act of) being bitten.
	\tr A cat---who bit a girl---caught fish.
\end{examples}

\begin{examples}
	\ex
	\label{ex:soldier_died}
	\script porohakin korsi pa arehedontak mae saeng
	\bits po- rohak -in korsi pa are- hedot -nak mae saeng
	\gloss PFV die AV soldier INST NMZ kill PV many person
	\tr A soldier had died, with the killing of many people.
	\tr A soldier---who killed many people---had died.
\end{examples}