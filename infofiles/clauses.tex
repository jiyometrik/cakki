\chapter{Clauses}
\label{ch:clauses}
This chapter discusses the basic clause structures of \langname{}.
It features \detail{symmetrical voice}, where the role of the primary argument
(or the argument \detail{in focus}) is encoded onto the verb itself. There are four voices in
\langname{}: \detail{actor voice} (\AV), \detail{patient voice} (\PV), \detail{locative voice} (\LV)
and \detail{instrumental voice} (\IV).

\section{Transitivity}
With \detail{transitive verbs}, all arguments are expressly marked with role markers.
The \detail{focus} --- the argument marked by the verb's voice --- receives the optional focus marker
\FOC~instead.
\begin{examples}
	\ex
	\label{ex:actor_voice}
	\script ngain wan yeo ri mosea \\
	\bits nga -in wan yeo ri mosea \\
	\gloss eat AV FOC child ACC crab \\
	\tr A child eats crab.
\end{examples}

\begin{examples}
	\ex
	\label{ex:patient_voice}
	\script nganak ce yeo wan mosea \\
	\bits nga -nak ce yeo wan mosea \\
	\gloss eat PV NOM child FOC crab \\
	\tr Crab is eaten by the child.
\end{examples}

\begin{examples}
	\ex
	\label{ex:locative_voice}
	\script ngaong ce yeo ri mosea wan sikaw \\
	\bits nga -ong ce yeo ri mosea wan sikaw \\
	\gloss eat LV NOM child ACC crab FOC hut \\
	\tr A hut is eaten crab in by a child.
\end{examples}

\begin{examples}
	\ex
	\label{ex:instrumental_voice}
	\script ngapay ri mosea ong sikaw ce yeo wan keroa \\
	\bits nga -pay ri mosea ong sikaw ce yeo wan ke= roa \\
	\gloss eat IV ACC crab LOC hut NOM child FOC 3SG.POS hand \\
	\tr A hut is eaten crab in by a child.
\end{examples}

\section{Intransitivity}
\detail{Intransitive verbs} don't require a voice, since they only take one argument.
\begin{examples}
	\ex
	\label{ex:the_child_dies}
	\script rohak wan yeo \\
	\bits rohak wan yeo \\
	\gloss die FOC child \\
	\tr A child dies.
\end{examples}

\begin{examples}
	\ex
	\label{ex:dead_child}
	\script porohak wan yeo \\
	\bits po- rohak wan yeo \\
	\gloss PFV die FOC child \\
	\tr A child is dead.
	\tr There is a dead child.
\end{examples}

This includes \detail{stative verbs} that describe a state of being.
\begin{examples}
	\ex
	\label{ex:akoe_beautiful}
	\script ngebande eno i Akoe  \\
	\bits nge- bande eno i Akoe  \\
	\gloss NPFV to.be.beautiful DEM.MED LNK PN \\
	\tr That Akoe is beautiful.
\end{examples}

\section{Nominal clauses}
\langname~lacks a copula; instead two nouns equating each other are placed side by side.
\begin{examples}
	\ex
	\label{ex:you_bitch}
	\script yebano \\
	\bits yeba =no \\
	\gloss dog 2SG.NOM \\
	\tr You're a dog.
\end{examples}
\begin{examples}
	\ex
	\label{ex:my_sister}
	\script Irade wan cisaono \\
	\bits Irade wan ci= saono  \\
	\gloss PN FOC 1SG.POS sister \\
	\tr Irade is my sister.
\end{examples}

\section{Relative clauses}
\subsection{Monotransitives}
The actor in a relative clause is often the same as in its surrounding independent clause.
\langname~uses the linker \LNK~to indicate this.
\begin{examples}
	\ex
	\label{ex:father_who_cried}
	\script sabowin i ando \\
	\bits sabow -in i ando \\
	\gloss cry AV LNK father \\
	\tr a crying father
\end{examples}
For monotransitive verbs it is useful to use \LOC~with \NMZ~too!
\begin{examples}
	\ex
	\label{ex:father_at_cried}
	\script doyatin ong aresabow wan ando \\
	\bits doyat -in ong are- sabow wan ando \\
	\gloss to.be.ill AV LOC NMZ cry FOC father \\
	\tr Ill is a father in (the process of) crying.
	\tr A crying father is ill.
\end{examples}

\subsection{Ditransitives}
The relative clause is squeezed in with the same role marker (\FOC, \NOM, and others)
as the argument it attaches to.
\begin{examples}
	\ex
	\label{ex:cat_bit_girl}
	\script podendain ri coecoe wan ea miw wan ingingkak boyan
	\bits po- denda -in ri coe^ coe wan ea miw wan ingik -nak boyan
	\gloss PFV catch AV ACC COL fish FOC DEM.PROX cat FOC bite PV girl
	\tr This cat who bit a girl caught fish.
\end{examples}
Alternatively, the combo of \INST~and \NMZ~also works, although this is a fair bit clunkier.
\begin{examples}
	\ex
	\label{ex:cat_who_bit_girl}
	\script podendain ea miw pa keareingingkak boyan ri coecoe \\
	\bits po- denda -in ea miw pa ke= are- ingik -nak boyan ri coe^ coe \\
	\gloss PFV catch AV DEM.PROX cat INST 3SG.POS NMZ bite PV girl ACC COL fish \\
	\tr A cat caught fish with a girl's (act of) being bitten.
	\tr A cat---who bit a girl---caught fish.
\end{examples}

\section{Negation}
\rom{Negatives} for verbal clauses are always formed using \rom{ihe}.
\begin{examples}
	\ex
	\label{ex:dont_eat_meat}
	\script ihe cinganak obaw \\
	\bits ihe ci= nga -nak obaw \\
	\gloss NEG 1SG.NOM eat PV flesh \\
	\tr I don't eat meat.
\end{examples}

\section{Imperatives}
\detail{Imperatives} are formed using the irrealis mood. Since the
actor is always implied, verbs in the imperative take \PV.
\begin{examples}
	\ex
	\label{ex:go_kill_goat}
	\script daehoak eno i wewin!
	\bits dae- hoak -nak eno i wewin
	\gloss IRR kill PV DEM.MED LNK goat
	\tr Kill that goat!
\end{examples}

\detail{Prohibitives} are formed with another particle, \rom{isop}.
\begin{examples}
	\ex
	\label{ex:dont_feed_that_child}
	\script isop daerahnganak yake i yeo
	\bits isop dae- rah- nga -nak yake i yeo
	\gloss NEG.IMP IRR CAUS eat PV DEM.DIST LNK child
	\tr Don't feed that child!
\end{examples}

% \section{Conditionals}
% The irrealis is also used to form \detail{conditionals}, statements in the hypothetical.
% Here, the dependent clauses don't take a voice---the main `resultative' clause does.
% \begin{examples}
% 	\ex
% 	\label{ex:pig_kills_me}
% 	\script mate cidasoh ea bao dain daakin ea aw cio
% 	\bits mate ci= dasoh ea bao dain dae- hoak -in ea aw cio
% 	\gloss if 1SG.NOM beat DEM.PROX pig later IRR kill AV DEM.PROX OBL 1SG
% 	\tr If I beat this pig, it'd kill me.
% \end{examples}

% \section{Questions}
% \detail{Yes/no questions} are formed by the interrogative mood.
% \begin{examples}
% 	\ex
% 	\label{ex:did_you_defecate_on_the_floor}
% 	\script we ponaengkinak maka ong bosin?
% 	\bits we po- na= engki -nak maka ong bosin
% 	\gloss INT PFV 2SG.NOM give PV shit LOC floor
% 	\tr Did you shit on the floor?
% \end{examples}

% \detail{{\scshape{Wh}}--questions} are formed by replacing the agent being asked
% about by a pronoun, with the corresponding voice on the verb.

% \begin{examples}
% 	\ex
% 	\label{ex:what_is_on_the_floor}
% 	\script masaikyo ya bosin?
% 	\bits ma- sai =kyo ya bosin
% 	\gloss VBZ.have what 3SG.POS POS floor
% 	\tr What does the floor have?
% 	\tr What's on the floor?
% \end{examples}

% \begin{examples}
% 	\ex
% 	\label{ex:what_time}
% 	\script ponaengkiong sodeng aw maka ri bosin?
% 	\bits po- na= engki -ong sodeng aw maka ri bosin
% 	\gloss PFV 2SG.NOM give LV when OBL shit LOC floor
% 	\tr When did you shit on the floor?
% \end{examples}

% \begin{examples}
% 	\ex
% 	\label{ex:with_shit}
% 	\script ponaengkipaw sai aw maka ong bosin?
% 	\bits po- na= engki -paw sai aw maka ong bosin
% 	\gloss PFV 2SG.NOM give IV what OBL shit LOC floor
% 	\tr For/with what did you shit on the floor?
% 	\tr Why/how did you shit on the floor?
% \end{examples}

% \begin{examples}
% 	\ex
% 	\label{ex:where_shit_you}
% 	\script ponaengkiong sian aw maka?
% 	\bits po- na= engki -ong sian aw maka
% 	\gloss PFV 2SG.NOM give LV where OBL shit
% 	\tr On where did you shit?
% \end{examples}

% \begin{examples}
% 	\ex
% 	\label{ex:who_shit}
% 	\script poengkiin soe aw maka ong bosin?
% 	\bits po- engki -in soe aw maka ong bosin
% 	\gloss PFV give AV who OBL shit LOC floor
% 	\tr Who shit on the floor?
% \end{examples}
