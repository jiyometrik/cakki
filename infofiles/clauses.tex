\chapter{Clause structure}

This chapter briefly discusses the syntactical structure of \langname{}.
It discusses also \langname{}'s case system, and its prioritisation of nouns'
roles in a clause.

% STICK WITH NOM-ACC ALIGNMENT
\section{Word order}
\langname{} is, by default, a subject--object--verb language.
This is exemplified in \cref{ex:pamat_hates_dog}. The subject of a transitive verb
is placed at the head of the sentence, then the direct object, then the verb itself.

\begin{example}
  \label{ex:pamat_hates_dog}
  \gloss
  Pamat & Pamat \\
  ta- & ACC \\
  pariw & dog \\
  yahin & hate \\
  =∅ & 3SG.NOM \\
  \tr Pamat hates (a certain) dog.
\end{example}

Yet, word order is shifted around very frequently. Usually, the head of the clause
is occupied by its \textit{focus}, the bit of information that the speaker wishes to emphasise
the most. The focus is followed by strings of additional information, usually
adhering to the same SOV order as before. Contrast \cref{ex:pamat_hates_dog} with the
following examples carrying the same meaning, \cref{ex:dog_pamat_hates,ex:hates_pamat_dog}.

\begin{example}
  \label{ex:dog_pamat_hates}
  \gloss
  ta- & ACC \\
  \highlight{pariw} & dog \\
  Pamat & Pamat \\
  yahin & hate \\
  =∅ & 3SG.NOM \\
  \tr A \textit{dog} is what Pamat hates.
\end{example}

\begin{example}
  \label{ex:hates_pamat_dog}
  \gloss
  \highlight{yahin} & hate \\
  =∅ & 3SG.NOM \\
  Pamat & Pamat \\
  ta- & ACC \\
  pariw & dog \\
  \tr \textit{To hate}, is what Pamat does to (a certain) dog.
\end{example}

\subsection{Postpositions}
Although later chapters will cover postpositions (along with converbs) in greater detail,
\textbf{postpositions} typically follow their attached noun. Observe \cref{ex:luan_kills_chicken_at_tree}:
the postposition \rom{ʔumul} (`below') comes \textit{after} the noun to which it is attached, \rom{raŋuy} (`willow').
\begin{example}
  \label{ex:luan_kills_chicken_at_tree}
  \gloss
  Luwan & Luan \\
  caʔi- & LOC \\
  \highlight{raŋuy} & willow \\
  \highlight{ʔumul} & below \\
  ta- & ACC \\
  carik & chicken \\
  ʔin- & PFV \\
  wahas & kill \\
  =∅ & 3SG.NOM \\
  \tr \textit{Luan} has killed a chicken below a willow tree.
\end{example}

\section{Stative verbs}
Stative verbs, in contrast to dynamic verbs, are intransitive
verbs without a direct object. In \langname{}, this is equivalent
to omitting the object role in a clause, where the default order is
SV (subject--verb). This is seen in \cref{ex:kawam_walking}, where the focus is the
subject, \rom{kawam} (a proper name).
\begin{example}
  \label{ex:kawam_walking}
  \gloss
  Kawam & Kawam \\
  pasi^ & PROG \\
  pasi & walk \\
  =∅ & 3SG.NOM \\
  \tr Kawam is walking.
\end{example}

As with transitive (or dynamic) verbs, the order in which
the subject and the verb occur can be reversed, so that the verb becomes the focus, as in
the equivalent \cref{ex:walking_kawam}.
\begin{example}
  \label{ex:walking_kawam}
  \gloss
  \highlight{pasi}^ & PROG \\
  \highlight{pasi} & walk \\
  =∅ & 3SG.NOM \\
  Kawam & Kawam \\
  \tr As for \textit{walking}, Kawam is doing so.
\end{example}

\section{Non-verbal clauses}
Non-verbal clauses are clauses in which a declarative verb (intransitive or transitive)
is not immediately apparent. This section illustrates how \langname{} handles such phrases.

\subsection{Existential clauses}
Existential clauses are clauses that, when translated to English, usually
begin with `there exist(s)' or `there are'. For such clauses,
\langname{} employs the intransitive verb \rom{semat} (`to stand') as an existential
verb, as in \cref{ex:theres_a_dog_here,ex:two_sens}.

\begin{example}
  \label{ex:theres_a_dog_here}
  \gloss
  pariw & dog \\
  ʔita & one \\
  kisak & CLF.small.animal \\
  caʔi- & LOC \\
  taŋamuy & here \\
  \highlight{semat} & stand \\
  =∅ & 3SG.NOM \\
  \tr There is one dog here.
\end{example}

\begin{example}
  \label{ex:two_sens}
  \gloss
  caʔi- & LOC \\
  Paŋkur & Bangur \\
  maruŋ & honorific.male \\
  Sen & Sen \\
  rawa & two \\
  ʔacak & CLF.human \\
  \highlight{semat} & stand \\
  halwat & 3PL \\
  ta- &  ACC \\
  Rupisen & Sen.Robi \\
  ni & and \\
  ta- & ACC \\
  Tiripsen & Sen.Dilip \\
  ʔamuy & DEM.PROX \\
  \lit In Bangur, two Mr Sen--s stand. They are that Robi Sen and that Dilip Sen.
  \tr There are two Sen--s in Bangur, Robi Sen and Dilip Sen.
  \source \textsc{5moyd} \#2021
\end{example}

\subsection{Locative clauses}
Locative clauses, in which the subject is \textit{in} a certain location, the
\acrfull{loc} is employed together with the `existential' verb \rom{semat} (`to stand'),
as in \cref{ex:mother_is_at_the_forest}.

\begin{example}
  \label{ex:mother_is_at_the_forest}
  \gloss
  ʔamu & mother \\
  caʔi- & LOC \\
  tentu^ & COL \\
  tentu & tree \\
  semat & stand \\
  =∅ & 3SG.NOM \\
  \tr Mother is in a forest.
\end{example}

\subsection{Nominal clauses}
Nominal clauses usually take the form `X is Y'. In \langname{}, this is indicated
by juxtaposing the two nominals, placing the subject (the noun being likened)
before the object (the noun being likened \textit{to}), since \langname{}
lacks copulae. See \cref{ex:he_is_pamat,ex:that_is_your_chicken}.
\begin{example}
  \label{ex:he_is_pamat}
  \gloss
  mera & 3SG \\
  ta- & ACC \\
  pamat & Pamat \\
  \tr He is Pamat.
\end{example}

\begin{example}
  \label{ex:that_is_your_chicken}
  \gloss
  ʔaliŋ & DEM.DIST \\
  ta- & ACC \\
  ka= & 2SG.GEN \\
  carik & chicken \\
  \tr That's your chicken.
\end{example}

Adjectival clauses, where Y in the `X is Y' format is an adjective, are handled
in the same manner. In this case, adjectives are treated as verbs, and thus
take on pronoun clitics, as in \cref{ex:khayat_is_tall,ex:my_clothes_were_beautiful}.
\begin{example}
  \label{ex:khayat_is_tall}
  \gloss
  Hayat & Khayat \\
  ŋela^ & PROG \\
  ŋelahu & long \\
  =∅ & 3SG.NOM \\
  \tr Hayat is (still) tall.
\end{example}

\begin{example}
  \label{ex:my_clothes_were_beautiful}
  \gloss
  se= & 1SG.GEN \\
  ʔanu- & NMLZ.PARG \\
  watis & wear \\
  puŋ- & PST \\
  culi^ & PROG \\
  culi & beautiful \\
  =wat & 3PL.NOM \\
  \tr My clothes were beautiful.
\end{example}

\section{Modifiers}
Modifiers are lexemes that supplement information to a base noun. This
section quickly discusses the order in which they occur after a noun.

\textbf{Adjectives} follow their attached noun, as in \cref{ex:white_dog}.
\begin{example}
  \label{ex:white_dog}
  % \script pariw mipu
  \gloss
  pariw & dog \\
  \highlight{purah} & white \\
  \tr (a) white dog
\end{example}

Similarly, \textbf{numbers} that detail the quantity of a certain referent also follow
it, as in \cref{ex:three_people}. In \cref{ex:six_books}, for instance,
the \textbf{noun classifier} also follows its attached referent, following the number.

\begin{example}
  \label{ex:three_people}
  % \script pariw ʔiker
  \gloss
  ʔacak & human \\
  \highlight{tami} & three \\
  \tr three people
\end{example}

\begin{example}
  \label{ex:six_books}
  \gloss
  huʔam & book \\
  halih & six \\
  \highlight{pelay} & CLF.sheet \\
  \tr six books
\end{example}

\textbf{\Acrfull{dem} nouns} also follow
their referents. Observe \cref{ex:these_two_trees,ex:that_black_dog} which demonstrate each
demonstrative noun combined with another modifier.

\begin{example}
  \label{ex:these_two_trees}
  % \script ʔamuy pariw ʔiker
  \gloss
  tentu & tree \\
  ŋelahu & long \\
  rawa & two \\
  waʔiŋ & CLF.tree \\
  \highlight{ʔamuy} & DEM.PROX \\
  \tr these two tall trees
\end{example}

\begin{example}
  \label{ex:that_black_dog}
  % \script ʔaliŋ pariw ʔaŋ wasuwasu
  \gloss
  pariw & dog \\
  tuʔat & black \\
  \highlight{ʔaliŋ} & DEM.DIST \\
  \tr that/yonder black dog
\end{example}

Contrastingly, \textbf{relative clauses} delimited by the \acrfull{rel} precede
their attached noun, as in \cref{ex:sleeping_dog}. More detail into relative
clauses will appear in future chapters.
\begin{example}
  \label{ex:sleeping_dog}
  \gloss
  wasu^ & PROG \\
  wasu & sleep \\
  \highlight{ʔaŋ} & REL \\
  \highlight{pariw} & dog \\
  \tr a dog which is sleeping
\end{example}

The \textbf{possessor} of the noun, which is marked in the \gls{gen}, also precedes
the noun. Since the `possessor' noun always takes on an additional case marker,
the `possession' can easily be distinguished from it, as in \cref{ex:kings_boar}.
\begin{example}
  \label{ex:kings_boar}
  \gloss
  \highlight{tapiwaŋa} & king \\
  ʔamuy & DEM.PROX \\
  me= & 3SG.GEN \\
  kataŋ & boar \\
  \tr this king's boar
\end{example}

Where the `possessor' is a personal pronoun, the personal pronoun is cliticised as a prefix onto the `possession',
as in \cref{ex:their_boar}.\sidenote{Case markers and pronoun clitics will be discussed in the following chapter!}
\begin{example}
  \label{ex:their_boar}
  \gloss
  \highlight{hal}= & 3PL.GEN \\
  kataŋ & boar \\
  mipu & red \\
  \tr their red boar
\end{example}

Where more than one possession occurs (i.e.~the possessors are `nested'),
each subsequent possessor is simply stacked before the previous one, as in
\cref{ex:his_grandmothers_doctors_dog}.
\begin{example}
  \label{ex:his_grandmothers_doctors_dog}
  \gloss
  me= & 3SG.GEN \\
  miwtiw & paternal.grandmother \\ % mawtaŋ - paternal.grandfather
  me= & 3SG.GEN \\
  ʔaca- & NMLZ.AARG \\ % agentive nominalizer
  ŋurus & drug \\
  me= & 3SG.GEN \\
  pariw & dog \\
  \tr his/her grandmother's doctor's dog
\end{example}

In short, nominal referents and their modifiers adopt the following general ordering.
\begin{tightcenter}
  relative clause -- possessor -- \textbf{referent} -- adjective -- number -- classifier -- demonstrative
\end{tightcenter}

\section{Adpositional phrases}
Adpositional phrases in \langname{} fall into one of three
`universal' categories: that of \textit{time}, \textit{manner} and \textit{place}.
Generally, the time--place--manner ordering of adpositional phrases is adopted, as
seen in \cref{ex:hayat_boil_millet}.
\begin{example}
  \label{ex:hayat_boil_millet}
  \gloss
  Hayat & Khayat \\
  ta- & ACC \\
  keʔiti & millet \\
  \highlight{nicawat} & tomorrow \\
  caʔi- & LOC \\
  \highlight{rewtew} & hut \\
  may- & INS \\
  \highlight{sakep} & pot \\
  kuŋi & boil \\
  =∅ & 3SG.NOM \\
  \tr Khayat will boil millet with a pot in a hut tomorrow.
\end{example}

Again, as with foci in a clause, these adpositional phrases may
be subject to reordering within it. An adpositional phrase can too
serve as the focus of a clause, as in the hypothetical \cref{ex:tomorrow_boil_millet}.

\begin{example}
  \label{ex:tomorrow_boil_millet}
  \gloss
  \highlight{nicawat} & tomorrow \\
  Hayat & Khayat \\
  ta- & ACC \\
  keʔiti & millet \\
  caʔi- & LOC \\
  \highlight{rewtew} & hut \\
  may- & INS \\
  \highlight{sakep} & pot \\
  kuŋi & boil \\
  =∅ & 3SG.NOM \\
  \tr \textit{Tomorrow}, Khayat will boil millet with a pot in a hut.
\end{example}
