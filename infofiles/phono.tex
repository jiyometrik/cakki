\chapter{Phonology}
\label{cha:phonology}

This chapter concerns the phonological aspects of \langname, including its
phonemes (\cref{sec:inventory}) and phonotactics (\cref{sec:phonotactics}). Notes
on allophonic variation (\cref{sec:allophonic_variation}) and the romanisation
system used throughout this reference grammar (\cref{sec:romanisation}) follow.


\section{Phonemic inventory}
\label{sec:inventory}
This section outlines the phonemes \langname{} contains, and briefly discusses
allophonic variation.

\subsection{Consonants}
\langname{} has a relatively small set of consonants, as detailed in the following
table (\cref{tab:consonants}). All symbols below can be found in the International Phonetic
Alphabet.

\begin{table}[htbp]
  \centering
  \begin{tabular}{r c c c c c}
    \toprule
    & \textbf{Labial} & \textbf{Alveolar} & \textbf{Palatal} & \textbf{Velar} & \textbf{Glottal} \\
    \midrule
    \textbf{Nasal} & m & n & & ŋ & \\
    \textbf{Plosive} & p & t & & k & ʔ \\
    \textbf{Affricate} & & t͡s & & & \\
    \textbf{Fricative} & & s & & x & \\
    \textbf{Tap} & & ɾ & & & \\
    \textbf{Approximant} & & l & j & w & \\
    \bottomrule
  \end{tabular}
  \caption{Consonant inventory of \langname{}}
  \label{tab:consonants}
\end{table}

As detailed in \cref{ex:consonants_initial}, all consonants can take initial positions in syllables.
Notions of syllables and their structure will be discussed later in \cref{sec:phonotactics}.
\begin{columns}
  \label{ex:consonants_initial}
  \cols \phomtext{m} & \phomtext{\highlight{m}əluk} & to fight
  \cols \phomtext{n} & \phomtext{\highlight{n}iniw} & breast
  \cols \phomtext{ŋ} & \phomtext{\highlight{ŋ}əʔ} & eye
  \cols \phomtext{p} & \phomtext{\highlight{p}istə} & to understand
  \cols \phomtext{t} & \phomtext{\highlight{t}əŋa} & to slit
  \cols \phomtext{k} & \phomtext{\highlight{k}əɾaʔ} & brittle
  \cols \phomtext{ʔ} & \phomtext{\highlight{ʔ}uli} & stomach
  \cols \phomtext{t͡s} & \phomtext{\highlight{t͡s}akki} & language
  \cols \phomtext{s} & \phomtext{\highlight{s}uxal} & to argue
  \cols \phomtext{x} & \phomtext{\highlight{x}aʔus} & to sigh
  \cols \phomtext{ɾ} & \phomtext{\highlight{ɾ}aŋuj} & willow
  \cols \phomtext{l} & \phomtext{\highlight{l}uwan} & male name
  \cols \phomtext{j} & \phomtext{\highlight{j}amu} & to be happy
  \cols \phomtext{w} & \phomtext{\highlight{w}atinak} & to accuse
\end{columns}

\subsection{Vowels}
\langname{} also has a relatively small inventory of vowels, represented in \cref{fig:vowels}.
It has five monophthongs largely spanning the entire vowel space, and no diphthongs.
\begin{figure}[htbp]
  \centering
  \begin{vowel}
    \vpoint{0}{3}{i}
    % \vpoint{1}{3}{i}
    \vpoint{2}{3}{u}
    \vpoint{1}{1.5}{ə}
    \vpoint{1}{0}{a}
  \end{vowel}
  \caption{Vowel inventory of \langname{}}
  \label{fig:vowels}
\end{figure}

Because few distinct vowel phonemes exist in the language, the realisations of
each vowel vary in different phonemic environments. However, neither length nor pitch
are distinguished in the language.

Some dialects of \langname{} may be starting to develop pitch accent,
but oral evidence for this is scarce.\sidenote{Do we want to add a tone system?}


\section{Phonotactics}
\label{sec:phonotactics}
This section outlines the syllable structure of \langname{}, before describing,
as its name aptly suggests, the language's phonotactics---how sounds are put together
to make `legal' words.

\subsection{Syllable structure}
Broadly, the syllable structure is CV(C). C represents any consonant, and V represents
any monophthong.

\Cref{ex:syllable_patterns} contains a selection of words with varying syllable patterns, up to
roots containing two syllables.
\begin{columns}
  \label{ex:syllable_patterns}
  \cols CV & \phomtext{ni} & and
  \cols CVC & \phomtext{ŋak} & to whine
  \cols CV.CV & \phomtext{pa.si} & to walk
  \cols CVC.CV & \phomtext{jak.ti} & red bean
  \cols CV.CVC & \phomtext{t͡sa.ʔut} & goat milk
  \cols CVC.CVC & \phomtext{ɾən.taw} & to laud
\end{columns}

Most \langname{} roots are disyllabic. Where partial or full reduplication
is applied, roots may extend to three or four syllables.

\subsection{Stress and prosody}
The above discussion on \detail{syllable} structure may be misleading---\langname{} is
a mora-timed language. The language obeys the following rules for determining
the number of morae a syllable contains.
\begin{itemize}
  \item The onset consonant does not contribute to the number of morae in the syllable, since every syllable must contain an onset.
  \item A syllable is \detail{monomoraic} if it is open (CV).
  \item A syllable is \detail{bimoraic} if it is closed (CVC).
\end{itemize}

Nominally, stress falls on the syllable which contains the penultimate mora.
To illustrate, \cref{tab:morae_patterns} contains the same selection of words as \cref{ex:syllable_patterns},
this time with the number of morae and the stress indicated.

\begin{table}[htpb]
  \centering
  \begin{tabular}{r r l l}
    \toprule
    \textbf{Syllable structure} & \textbf{Morae} & \textbf{Stress pattern} & \textbf{Gloss} \\
    \midrule
    CV & 1 & \phomtext{ˈni} & and \\
    CVC & 2 & \phomtext{ˈŋak} & to whine \\
    CV.CV & 1--1 & \phomtext{ˈpa.si} & to walk \\
    CVC.CV & 2--1 & \phomtext{ˈjak.ti} & red bean \\
    CV.CVC & 1--2 & \phomtext{t͡sa.ˈʔut} & goat milk \\
    CVC.CVC & 2--2 & \phomtext{ɾən.ˈtaw} & to laud \\
    \bottomrule
  \end{tabular}
  \caption{Morae patterns in disyllabic roots}
  \label{tab:morae_patterns}
\end{table}

\subsection{Phonotactics---within the syllable}
Below are general phonotactic rules governing the structure of individual \langname{} syllables.
\begin{itemize}
  \item A syllable cannot both start and end with \phomtext{ʔ}.
  \item The combinations \phomtext{ji ji wu ij ij uw} are disallowed.
\end{itemize}

Curiously, a pseudo--vowel harmony system may be starting to emerge.
\langname{} roots seldom contain \phomtext{i}
and the other high vowels \phomtext{i u} together. Where \phomtext{i} is found,
the central vowels \phomtext{ə a} are likely to coexist within the same root instead.
However, all other vowels may coexist in the same word. Therefore, the words
\phomtext{ʔiɾə} and \phomtext{ʔiɾa} are likely to exist, much more so than
the words \phomtext{ʔiɾi} and \phomtext{ʔiɾu}.\sidenote{This may be simply speculation, since \phomtext{i} is the least commonly occurring monophthong.}

\subsection{Phonotactics---between syllables}
The consonant sequence \phomtext{ʔʔ} is disallowed\sidenote{Pronouncing this would be a nightmare!}.
% Between syllables, \phomtext{ʔ} may only be placed between two vowels.
% \phomtext{Cʔ} and \phomtext{ʔC} sequences are disallowed, including \phomtext{ʔʔ}\sidenote{Pronouncing this would be a nightmare!}.

Only the plosive--affricate sequences outlined in \cref{tab:stop_affricate} are allowed.
Each row represents the first consonant in the cluster; each column represents the second consonant in the cluster, hence
\phomtext{ʔə\highlight{p.t͡s}ə} is an allowed cluster, whereas \phomtext{ʔə\highlight{t͡s.p}ə} is not.
\begin{table}[htbp]
  \centering
  \begin{tabular}{c c c c c}
    \toprule
    & p & t & t͡s & k \\
    \midrule
    p & + & + & + & + \\
    t & -- & + & -- & -- \\
    t͡s & -- & -- & + & -- \\
    k & -- & + & + & + \\
    \bottomrule
  \end{tabular}
  \caption{Allowed plosive--affricate sequences}
  \label{tab:stop_affricate}
\end{table}

Fricatives and approximants may cluster with plosives (excluding \phomtext{ʔ}), fricatives
and approximants between two syllables.

Where a nasal and an obstruent cluster, the nasal should match its place of articulation.
To illustrate, \phomtext{ʔə\highlight{m.p}ə} would be a valid word, but not
\phomtext{ʔə\highlight{m.t͡s}ə} or \phomtext{ʔə\highlight{n.p}ə}.


\section{Allophonic variation}
\label{sec:allophonic_variation}

Due to the rather minimalistic phonology of \langname, allophonic variation
is rife in various phonological environments. We discuss allophonic variation
in consonants and vowels respectively.

\subsection{Plosives}
The voiceless plosives \phomtext{p t k} (except \phomtext{ʔ}) become voiced commonly after nasal consonants, as in \cref{ex:plosives_middle}.
No minimal pairs between plosives \phomtext{p t k} and their voiced allophones \phontext{b d ɡ} exist, however.
\begin{columns}[cols.markup=\mutations]
  \label{ex:plosives_middle}
  \cols \phomtext{ʔum\highlight{p}uj} & \phontext{ʔʊm.\highlight{b}ʊj} & birch
  \cols \phomtext{tən\highlight{t}u} & \phontext{tən.\highlight{d}u} & rock
  \cols \phomtext{xaŋ\highlight{k}a} & \phontext{xaŋ.\highlight{ɡ}a} & firewood
\end{columns}

Plosives in syllable-final positions (except \phomtext{ʔ})
become unreleased, as in \cref{ex:plosives_final}. In particular, syllable-final
\phomtext{k} is often realised as
\phontext{ʔ} due to their relative proximity in the oral cavity.

\begin{columns}[cols.markup=\mutations]
  \label{ex:plosives_final}
  \cols \phomtext{saɾa\highlight{p}} & \phontext{sa.ɾa\highlight{p̚}} & dawn
  \cols \phomtext{naŋki\highlight{t}} & \phontext{naŋ.ɡɪ\highlight{t̚}} & to feed (livestock)
  \cols \phomtext{ʔali\highlight{k}} & \phontext{ʔa.lɪ\highlight{ʔ}} & daikon
  \cols \phomtext{ŋu\highlight{k}tu} & \phontext{ŋʊ\highlight{ʔ}.tu} & to pound
\end{columns}

\subsection{\phontext{x\allo h\allo ɦ} alternation}
Due to its proximity to the glottis, the velar fricative \phomtext{x} may be
realised as \phontext{h} in all environments for convenience. Fairly widespread
among younger speakers, a \phomtext{x} > \phomtext{h} sound shift may be possible.

More interestingly, however, \phontext{x\allo h} may be further lenited to \phontext{ɦ\allo ∅}
intervocalically. Vowel hiatus may arise because of this, violating the general rule that
all syllables must begin with a consonant, as in \cref{ex:hiatus}.

\begin{columns}[cols.markup=\mutations]
  \label{ex:hiatus}
  \cols \phomtext{ja\highlight{x}in} & \phontext{ja\highlight{.}in} & to hate
  \cols \phomtext{mina\highlight{x}aw} & \phontext{mi.na\highlight{.}aw\allo mi.na\highlight{ː}w} & bamboo
\end{columns}

\subsection{\phontext{ɾ\allo d} alternation}
The tap \phomtext{ɾ} may sometimes be realised as \phontext{d}, or even \phontext{ð}
in consonant clusters (where a closed syllable meets another syllable starting with a consonant)
and syllable-finally, as in \cref{ex:r_d}.
This does not clash with \phomtext{t} being realised as \phontext{d} in voiced contexts.

\begin{columns}[cols.markup=\mutations]
  \label{ex:r_d}
  \cols \phomtext{xu\highlight{ɾ}na} & \phontext{xʊ\highlight{d}.na\allo xʊ\highlight{ð}.na} & moon
  \cols \phomtext{nawa\highlight{ɾ}} & \phontext{na.wa\highlight{d}} & to greet
\end{columns}

\subsection{\phomtext{l}--vocalisation}
The lateral approximant \phomtext{l} can usually be alternated with \phontext{w\allo ʊ̯},
as in \cref{ex:l_vocalisation}. This happens most frequently syllable-finally, yet never
happens syllable-initially. This occurrence is fairly common among younger speakers, although it
is considered improper.
\begin{columns}[cols.markup=\mutations]
  \label{ex:l_vocalisation}
  \cols \phomtext{ʔəni\highlight{l}} & \phontext{ʔə.nɪ\highlight{ʊ̯}} & aubergine
  \cols \phomtext{mə\highlight{l}uk} & \phontext{mə.\highlight{l}uʔ\allo mə.\highlight{w}uʔ} & to fight
\end{columns}

\subsection{Vowels in closed syllables}
As shown amply in the above examples, the monophthongs \phontext{i u} tend to
weaken or have their tongue root positions retracted in closed syllables, giving
approximately \phomtext{ɪ ʊ}, seen in \cref{ex:vowel_weakening}.
\begin{columns}[cols.markup=\mutations]
  \label{ex:vowel_weakening}
  \cols \phomtext{nin\highlight{i}w} & \phontext{ni.n\highlight{ɪ}w} & breast
  \cols \phomtext{ʔal\highlight{i}k} & \phontext{ʔa.l\highlight{ɪ}ʔ} & daikon
  \cols \phomtext{ʔ\highlight{u}mp\highlight{u}j} & \phontext{ʔ\highlight{ʊ}m.b\highlight{ʊ}j} & birch
\end{columns}
In particularly divergent dialects, the combinations \phomtext{iw uj} may even
yield \phontext{ew oj}.\sidenote{Perhaps this also signifies the first step towards the monophthongisation of vowel--approximant sequences!}


\section{Romanisation}
\label{sec:romanisation}

\Cref{tab:consonants_romanised} shows the romanised versions of each consonant.
For clarity, each consonant phoneme is spelt with only one character. In addition,
the following stylistic choices have been made:
\begin{itemize}
  \item Writing \phomtext{t͡s} as \rom{ts} may cause confusion with \phomtext{t.s}, so \rom{c} is used instead.
  \item Writing \phomtext{ŋ} as \rom{ng} may cause confusion with \phomtext{ŋ.ɡ}, so \rom{ŋ} is used instead.
  \item Writing \phomtext{ɾ} as its IPA symbol is cumbersome, so \rom{r} is used instead.
\end{itemize}
The romanised consonants are shown in \cref{tab:consonants_romanised}.
\begin{table}[htbp]
  \centering
  \begin{tabular}{r c c c c c}
    \toprule
    & \textbf{Labial} & \textbf{Alveolar} & \textbf{Palatal} & \textbf{Velar} & \textbf{Glottal} \\
    \midrule
    \textbf{Nasal} & \rom{m} & \rom{n} & & \rom{ŋ} & \\
    \textbf{Plosive} & \rom{p} & \rom{t} & & \rom{k} & \rom{ʔ} \\
    \textbf{Affricate} & & \rom{c} & & & \\
    \textbf{Fricative} & & \rom{s} & & \rom{h} & \\
    \textbf{Tap} & & \rom{r} & & & \\
    \textbf{Approximant} & & \rom{l} & \rom{y} & \rom{w} & \\
    \bottomrule
  \end{tabular}
  \caption{Romanised consonants in \langname{}}
  \label{tab:consonants_romanised}
\end{table}

The vowels will be romanised as in \cref{fig:vowels_romanised}. Due to the relative
difficulty in typing the letter \rom{ə}, it will be romanised with \rom{e} instead.
All other monophthongs remain as they are in the IPA.
\begin{figure}[htbp]
  \centering
  \begin{vowel}
    \vpoint{0}{3}{\rom{i}}
    % \vpoint{1}{3}{\rom{i}}
    \vpoint{2}{3}{\rom{u}}
    \vpoint{1}{1.5}{\rom{e}}
    \vpoint{1}{0}{\rom{a}}
  \end{vowel}
  \caption{Romanised vowels in \langname{}}
  \label{fig:vowels_romanised}
\end{figure}

Every phoneme will be spelt out in its original form: no allophony will be indicated, and glottal stops
will always be written. Since stress is non-contrastive, stress will not be indicated.
