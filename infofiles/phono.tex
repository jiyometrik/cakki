\chapter{Phonology}
\label{cha:phonology}

Rather self-explanatory chapter title, don't you think?
This chapter describes \langname's \detail{phonology}.

\section{Consonants}
\label{sec:consonants}

\langname~has a modestly-sized consonant inventory---\cref{tab:consonants}. Each column
shows the \detail{lenition pattern} of each broad place of articulation---see \cref{sec:lenition}.
\begin{table}[htpb]
	\centering
	\begin{tabular}{@{}rlllll@{}}
		\toprule
		                       & labial                 & alveolar               & sibilant              & palatal               & dorsal                \\
		\midrule
		plosive \textit{tense} & pː\hfill\orthtext{p'}  & tː\hfill\orthtext{t'}  & sː\hfill\orthtext{s'} & cː\hfill\orthtext{c'} & kː\hfill\orthtext{k'} \\
		plosive \textit{plain} & p\hfill\orthtext{p}    & t\hfill\orthtext{t}    & s\hfill\orthtext{s}   & c\hfill\orthtext{c}   & k\hfill\orthtext{k}   \\
		fricative              & f\hfill\orthtext{f}    & ɬ\hfill\orthtext{ll}   &                       & ɕ\hfill\orthtext{x}   & h\hfill\orthtext{h}   \\
		plosive \textit{soft}  & b\hfill\orthtext{b\;m} & d\hfill\orthtext{d\;n} &                       & ɟ\hfill\orthtext{j ñ} & ɡ\hfill\orthtext{g}   \\
		liquid                 & ʋ\hfill\orthtext{v}    & l\hfill\orthtext{l}    &                       & j\hfill\orthtext{y}   &                       \\
		\bottomrule
	\end{tabular}
	\caption{Consonant inventory}
	\label{tab:consonants}
\end{table}

\begin{itemize}
	\item \phomtext{pː p b} are bilabial; \phomtext{f ʋ} are labiodental
	\item \phomtext{tː t ɬ d l} are dental; \phomtext{sː s} are apico-alveolar
	\item \phomtext{cː c ɟ ɕ} are alveolo-palatal; \phomtext{j} is true palatal
	\item \phomtext{kː k ɡ} are velar
	\item \phomtext{h} can be realised as any voiceless `back' fricative (velar, uvular, pharyngeal, glottal)
\end{itemize}

\subsection{On \phomtext{s sː}}
The true fricatives \phomtext{s sː} are analysed as `plosives' in \cref{tab:consonants},
as they were merged with with older \phomtext{ts tːs}. Previously, the sibilant affricates also complied with
the lenition pattern in \cref{sec:lenition} to form \phontext{s}, but they lenited themselves to \phomtext{s sː}.

\subsection{Consonant taxonomy}
The `tense' plosives \phomtext{pː tː sː cː kː} are realised as long consonant
geminates word-medially and word-finally, and as tenuis \phontext{p͈ t͈ s͈ c͈ k͈} word-initially.

The `soft' plosives \phomtext{b d ɟ ɡ} are realised as voiced \phontext{b d ɟ ɡ} word-initially,
prenasalised \phontext{ᵐb ⁿd ᶮɟ ᵑɡ} word-medially, and as fully nasal \phontext{m n ɲ ŋ}
word-finally and adjacent to nasal vowels.

\section{Vowels}
\label{sec:vowels}

\langname~has comparatively few vowels---\cref{fig:vowels}. All vowels have
nasalised counterparts, doubling the total number of monophthongs.

\begin{figure}[htpb]
	\centering
	\begin{vowel}
		\vpoint{0}{3}{i}
		\vpoint{0.25}{2.5}{ĩ}
		\vpoint{0}{1.5}{e}
		\vpoint{0.25}{1.1}{ẽ}
		\vpoint{2}{1.5}{o}
		\vpoint{1.75}{1.9}{õ}
		\vpoint{1}{0}{a}
		\vpoint{1.65}{.5}{ã}
	\end{vowel}
	\caption{Vowel inventory}
	\label{fig:vowels}
\end{figure}

\subsection{Vowel taxonomy}
\phomtext{i ĩ o õ} are realised as \phontext{ɨ ɨ̃ u ũ} adjacent to alveolar and dorsal consonants;
they retain their original qualities adjacent to palatal consonants.

Oral vowels can only be in hiatus with other oral vowels, and nasal vowels with other nasal vowels.
Ergo, affixes that cause a vowel to be adjacent to a vowel of the opposite nasality
will change their own vowel to match the nasality of the adjacent vowel.

\section{Tone}
\label{sec:tone}

\detail{Tone} is a syllable-level feature. In \langname, syllables carry one of two tones,
a \detail{high tone} \phomtext{◌́} or a \detail{zero tone} \phomtext{◌}, the latter of which surfaces as mid.

Similarly to vowel nasality, where two vowels of opposite tone are adjacent,
the latter vowel assimilates the tone of the former vowel.

\section{Phonotactics}
\label{sec:phonotactics}

\langname~organises its phonemes rather modestly.
The structure of its phonological word is as follows\sidenote{The notation used here is \detail{Recursive Baerian Phonotactics Notation}. \href{https://llblumire.github.io/recursive-baerian-phonotactics-notation/20170801_Recursive_Baerian_Phonotactics_Notation.pdf}{link}.}.
\[
	\large
	\#
	\bigg[_{\hspace{0.2em}\omega}
	% \text{C}_0^{?}
	\Big[_{\hspace{0.1em}\sigma}
	\text{T} \text{C}_1^{?} \text{V} \text{C}_2^{?}
	\Big]
	\sigma^{*}
	\bigg]
	\#
\]

\begin{itemize}
	\item \(\#\) is a word boundary; \([\:]\) a domain
	\item \(\omega\) is a phonological word; \(\sigma\) a syllable
	\item \({}^{?}\) means `zero or one'; \({}^*\) means `zero or more times'
	\item \(\text{T}\) is a tone; \(\text{C}_{1-2}\) are consonants; \(\text{V}\) is a vowel, nasalised or plain
\end{itemize}

\langname~has a (C)V(C) structure. It allows consecutive vowels (including the same vowel appearing twice) in hiatus,
although vowels in hiatus must match in nasality and tone---\cref{sec:vowels,sec:tone}.

There are some restrictions on consonant clusters \(\text{C}_\alpha\text{C}_\beta\) between syllables:
\begin{itemize}
	\item \(\text{C}_\alpha \neq \text{C}_\beta\); ergo, geminates cannot occur
	      \begin{itemize}
		      \item Geminate plosives \phomtext{p.p t.t s.s c.c k.k} are realised as tense plosives
		      \item Geminate non-plosives are simplified to their non-geminated forms
	      \end{itemize}
	\item The lenition level of \(\text{C}_\beta\) must be less than or equal to that of \(\text{C}_\alpha\), i.e. \(\text{C}_\alpha\) must be of higher sonority---\cref{sec:lenition}
\end{itemize}

\section{Lenition}
\label{sec:lenition}

\detail{Lenition} (denoted by \orthtext{ᴸ}) is a process that takes place
semi-pervasively in \langname~especially across morpheme boundaries, where it is
used to simplify otherwise `tricky' consonant clusters. Consonants are organised
into \detail{levels} of lenition (represented by \(\lambda\)), where \({-1}\)
and \(3\) represent the most fortis and most lenis forms---\cref{tab:lenition_consonants}.

\begin{table}[htpb]
	\centering
	\begin{tabular}{@{}rccccc@{}}
		\toprule
		\(\lambda\) & labial   & alveolar & sibilant & palatal  & dorsal   \\
		\midrule
		\(-1\)      & \rom{p'} & \rom{t'} & \rom{s'} & \rom{c'} & \rom{k'} \\
		\(0\)       & \rom{p}  & \rom{t}  & \rom{s}  & \rom{c}  & \rom{k}  \\
		\(1\)       & \rom{f}  & \rom{ll} &          & \rom{x}  & \rom{h}  \\
		\(2\)       & \rom{b}  & \rom{d}  &          & \rom{j}  & \rom{g}  \\
		\(3\)       & \rom{v}  & \rom{l}  &          & \rom{y}  & \rom{∅}  \\
		\bottomrule
	\end{tabular}
	\caption{Lenition patterns}
	\label{tab:lenition_consonants}
\end{table}

\phomtext{s} is impervious to further lenition; it spans levels \(0\) to \(3\).

When a consonant--consonant cluster \(\text{C}_\alpha\text{C}_\beta\) is formed
by affixes, both consonants lenite by one level, forming \(\text{G}_\alpha\text{G}_\beta\).
The usual phonotactic rules apply---\cref{sec:phonotactics}.

If \(\text{G}_\alpha\) is of \textit{lower} sonority than \(\text{G}_\beta\),
\(\text{G}_\alpha\) and \(\text{G}_\beta\) metathesise.