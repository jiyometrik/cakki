\chapter{Phonology}
\label{cha:phonology}

Rather self-explanatory chapter title, don't you think?
This chapter describes \langname's \detail{phonology}.

\section{Consonants}
\label{sec:consonants}

\langname~has a modestly-sized consonant inventory---\cref{tab:consonants}. Each column
shows the \detail{lenition pattern} of each broad place of articulation---see \cref{sec:lenition}.
\begin{table}[htpb]
	\centering
	\begin{tabular}{@{}rccccc@{}}
		\toprule
		                       & labial & alveolar & sibilant & palatal & dorsal \\
		\midrule
		nasal                  & m      & n        &          & ɲ       & ŋ      \\
		plosive \textit{tense} & pː     & tː       & sː       & cː      & kː     \\
		plosive \textit{plain} & p      & t        &          & c       & k      \\
		fricative              &        & ɬ        & s        & ɕ       & h      \\
		liquid                 & ʋ      & l        & r\allo z & j       &        \\
		\bottomrule
	\end{tabular}
	\caption{Consonant inventory}
	\label{tab:consonants}
\end{table}

\begin{itemize}
	\item \phomtext{m pː p} are bilabial % though m is often realised as ~w
	\item \phomtext{ʋ} is labiodental
	\item \phomtext{n tː t ɬ l} are dental; \phomtext{sː s} are apico-alveolar
	\item \phomtext{ɲ cː c ɕ} are alveolo-palatal; \phomtext{j} is true palatal
	\item \phomtext{ŋ kː k} are velar, though \phomtext{ŋ} is often realised as \phontext{ɰ̃}
	\item \phomtext{h} can be realised as any voiceless `back' fricative (velar, uvular, pharyngeal, glottal)
	\item standalone \phomtext{r} is often fricated: alveolar \phontext{z} or palato-alveolar \phontext{r̝}; in consonant clusters it is \phontext{ɾ}
\end{itemize}

\subsection{On \phomtext{s sː}}
The true fricatives \phomtext{s sː} are analysed as `plosives' in \cref{tab:consonants},
as they were merged with with older \phomtext{ts tːs}. Previously, the sibilant affricates also complied with
the lenition pattern in \cref{sec:lenition} to form \phontext{s}, but they lenited themselves to \phomtext{s sː}.

\subsection{Consonant taxonomy}
The `tense' plosives \phomtext{pː tː sː cː kː} are realised as long consonant
geminates word-medially and word-finally, and as tenuis \phontext{p͈ t͈ s͈ c͈ k͈} word-initially.

Voiceless obstruents can become voiced after voiced segments (including vowels). In particular,
the plain plosives \phomtext{p t s c k} are prone to becoming prenasalised \phontext{ᵐb ⁿd ⁿs ᶮɟ ᵑɡ},
especially after nasal vowels.

\section{Vowels}
\label{sec:vowels}

\langname~has comparatively few vowels---\cref{fig:vowels}.

\begin{figure}
	\centering
	\begin{vowel}
		\vpoint{0}{3}{i}
		\vpoint{0.25}{2.5}{ĩ}
		\vpoint{1}{2.5}{ɨ}
		\vpoint{0.75}{2.1}{ɨ̃}
		\vpoint{0}{1.5}{e}
		\vpoint{0.25}{1.25}{ẽ}
		\vpoint{2}{1.5}{o}
		\vpoint{1.75}{1.75}{õ}
		\vpoint{1}{0}{a}
		\vpoint{1.5}{.5}{ã}
	\end{vowel}
	\caption{Vowel inventory}
	\label{fig:vowels}
\end{figure}

\langname~allows any sequence of consecutive vowels (in separate syllables), including
the same vowel doubled. Where a nasal and an oral vowel appear adjacent to each other,
the sequence is completely nasal.

\subsection{Vowel taxonomy}
\phomtext{i ĩ o õ} are realised as \phontext{ɨj ɨ̃j u ũ} adjacent to alveolar and velar consonants,
they retain their original qualities adjacent to palatal consonants.

The vowel denoted \phomtext{ɨ} (and its nasal counterpart) stands for a generally
high-mid, central vowel. Its realisation varies between and within speakers as \phontext{ɨ\allo ɘ\allo ə\allo ɵ}.


\section{Romanisation system}
\label{sec:romanisation}

The vowels \phomtext{i ɨ e a o} are romanised as \orthtext{i y e a o}, and their
nasal counterparts are \orthtext{ĩ ỹ ẽ ã õ}. Where two vowels appear adjacent to
each other, they are written consecutively---\orthtext{ỹã} is \phomtext{ɨ̃ã}.

Some of the choices in romanising consonants are for aesthetic purposes, but the
romanisation system remains largely unambiguous--\cref{tab:consonants_romanised}.

\begin{table}[htpb]
	\centering
	\begin{tabular}{@{}rccccc@{}}
		\toprule
		                       & labial      & alveolar    & sibilant    & palatal       & dorsal        \\
		\midrule
		nasal                  & \native{m}  & \native{n}  &             & \native{ñ}    & \native{g ng} \\
		plosive \textit{tense} & \native{p'} & \native{t'} & \native{s'} & \native{c'}   & \native{k'}   \\
		plosive \textit{plain} & \native{p}  & \native{t}  & \native{s}  & \native{c}    & \native{k}    \\
		fricative              & \native{v}  & \native{z}  & \native{s}  & \native{x ç}  & \native{h}    \\
		liquid                 &             & \native{l}  & \native{r}  & \native{j ll} &               \\
		\bottomrule
	\end{tabular}
	\caption{Consonant romanisation}
	\label{tab:consonants_romanised}
\end{table}

\begin{itemize}
	\item \phomtext{ŋ} is \orthtext{ng} in consonant clusters, and \orthtext{g} otherwise
	\item \phomtext{ɕ} is \orthtext{ç} in consonant clusters, and \orthtext{x} otherwise
	\item \phomtext{j} is \orthtext{ll} syllable-finally, and \orthtext{j} otherwise
\end{itemize}

\section{Phonotactics}
\label{sec:phonotactics}

\langname~organises its phonemes rather modestly.
The structure of its phonological word is as follows\sidenote{The notation used here is \detail{Recursive Baerian Phonotactics Notation}. \href{https://llblumire.github.io/recursive-baerian-phonotactics-notation/20170801_Recursive_Baerian_Phonotactics_Notation.pdf}{link}.}.
\[
	\large
	\#
	\bigg[_{\hspace{0.2em}\omega}
	% \text{C}_0^{?}
	\Big[_{\hspace{0.1em}\sigma}
	\text{C}_1^{?} \text{V} \text{C}_2^{?}
	\Big]
	\sigma^{*}
	\bigg]
	\#
\]

\begin{itemize}
	\item \(\#\) is a word boundary; \([\:]\) a domain
	\item \(\omega\) is a phonological word; \(\sigma\) a syllable
	\item \({}^{?}\) means `zero or one'; \({}^*\) means `zero or more times'
	\item \(\text{C}_{1-2}\) are consonants; \(\text{V}\) is a vowel, nasalised or plain
\end{itemize}

\langname~has a (C)V(C) structure. It allows consecutive vowels (including the same vowel appearing twice) in hiatus;
if a plain and nasalised vowel appear adjacent to each other, they are both nasalised.

There are some restrictions on consonant clusters between syllables:
\begin{itemize}
	\item Non-plosive geminates are disallowed, so \phomtext{m.m} simplifies to \phomtext{m}, and \phomtext{r.r} to \phomtext{r}
	      \begin{itemize}
		      \item Plosive geminates automatically simplify to their tenius versions
	      \end{itemize}
	\item Tenius plosives cannot form clusters
	\item The second consonant in the cluster cannot have a higher lenition level than the first consonant
\end{itemize}

\section{Lenition}
\label{sec:lenition}

\detail{Lenition} (denoted by \orthtext{ᴸ}) is a process that takes place
semi-pervasively in \langname. Consonants are organised into \detail{levels} of
lenition, where --1 and 3 represent the most fortis and most lenis variants---\cref{tab:lenition_consonants}.

\begin{table}[htpb]
	\centering
	\begin{tabular}{@{}rccccc@{}}
		\toprule
		level & labial      & alveolar    & sibilant    & palatal     & dorsal      \\
		\midrule
		--1   & \native{p'} & \native{t'} & \native{s'} & \native{c'} & \native{k'} \\
		0     & \native{p}  & \native{t}  & \native{s}  & \native{c}  & \native{k}  \\
		1     & \native{v}  & \native{z}  & \native{r}  & \native{x}  & \native{h}  \\
		2     & \native{m}  & \native{l}  & \native{l}  & \native{j}  & \native{g}  \\
		3     & \native{m}  & \native{n}  & \native{n}  & \native{ñ}  & \rom{∅}     \\
		\bottomrule
	\end{tabular}
	\caption{Lenition patterns}
	\label{tab:lenition_consonants}
\end{table}

As mentioned in \cref{sec:consonants}, \phomtext{sː s} give \phomtext{s} in lenition environments
because they descend from ancestral affricates \phomtext{*tːs *ts} that had complied with this lenition pattern.
In a similar vein, \phomtext{t} previously lenited to \phomtext{*θ} and \phomtext{*tɬ} to \phomtext{ɬ}; however \phomtext{*tɬ} later merged with \phomtext{t} and \phomtext{*θ} with \phomtext{ɬ}.

The process of lenition takes place in some morphological environments
as a means to simplify otherwise `tricky' consonant clusters. When a consonant--consonant
cluster \(\text{C}_a\text{C}_b\) is created across a morpheme boundary, some rules govern its behaviour:
\begin{itemize}
	\item If \(\text{C}_a\) is a nasal (lenition level 3), it matches the place of articulation of the lenited \({C}_b\) --- \phomtext{n.k} lenites to \phomtext{*n.h}, which then simplifies to \phomtext{ŋ.h}
	\item If the lenited version of \(\text{C}_b\) is \phomtext{r}, it metathesises with lenited \(\text{C}_a\) --- \phomtext{p.s} lenites to \phomtext{*v.r}, which then metathesises to \phomtext{r.ʋ}
\end{itemize}

%% what if? /j/ - g, /s\/ - j 
Lenition can take place in compound words:
\begin{columns}[cols.markup=\mutations]
	\cols \rom{mẽ·tok} & \rom{mẽzok} & mom--dad, parent
	\cols \rom{hoc·ser} & \rom{hollrer} & wing--shark, shark's fin
	\cols \rom{ñe·coan} & \rom{ñejoan} & two thousand
\end{columns}

And it also arises as a result of the oblique case:

\begin{columns}[cols.markup=\mutations]
	\cols \native{pĩᴸ-cal} & \rom{pĩxal} & person (\OBL)
	\cols \native{eᴸ-võã}  & \rom{emõã} & dog (\OBL)
	\cols \native{ᴸ-kat'o}  & \rom{hat'o} & rock (\OBL)
\end{columns}

\subsection{What about the opposite?}
\detail{Fortition} (the opposite of lenition), denoted with \orthtext{ᴷ}, can happen in the imperative mood:
\begin{columns}[cols.markup=\mutations]
	\cols \native{aᴷ-top'} & \rom{at'op'} & jump (\IMP)!
	\cols \native{aᴷ-sẽĩr}  & \rom{as'ẽĩr} & respect (\IMP)!
	\cols \native{aᴷ-anĩx} & \rom{aganĩx} & remember (\IMP)!
	% \cols \native{ᴸ-kato}  & \rom{hato} & rock (\OBL)
\end{columns}