\chapter{Phonology}
\label{cha:phonology}
Rather self-explanatory chapter title, don't you think?


\section{Consonants}

\langname{} has a medium-sized set of consonants---\cref{tab:consonants}.

\begin{table}[htbp]
  \begin{tabular}{r c c c c}
    \toprule
                & {labial} & {alveolar}       & {palatal}      & {velar}         \\
    \midrule
    {nasal}     & m        & n                &                & ŋ \orthtext{ng} \\
    {plosive}   & p\;b     & t\;d             &                & k               \\
    {affricate} &          & t͡s \orthtext{ts} &                &                 \\
    {fricative} &          & s                &                & x \orthtext{h}  \\
    {liquid}    &          & ɾ \orthtext{r}   & j \orthtext{y} & w               \\
    \bottomrule
  \end{tabular}
  \caption{Consonant inventory}
  \label{tab:consonants}
\end{table}

The intervocalic sequences \phomtext{t.s t͡s} and \phomtext{t͡sː t.t͡s} have no minimal pairs.

The velar nasal \phomtext{ŋ} is able to nasalise the preceeding vowel syllable-finally, e.g. \phomtext{oŋ} becomes \phontext{õ\allo õɰ̃\allo õŋ}.
The voiced plosives \phomtext{b d} are softened \phontext{v ɾ} intervocalically and fortified \phontext{p t} syllable-finally.
The velar fricative \phomtext{x} can also be realised \phontext{ɰ\allo h\allo ɦ}.

\section{Vowels}

\langname{} also has a relatively small inventory of monophthongs---\cref{tab:vowels}.

\begin{table}[htbp]
  \begin{tabular}{r c c}
    \toprule
         & front & non-front \\
    \midrule
    high & i     & o         \\
    low  & e     & a         \\
    \bottomrule
  \end{tabular}
  \caption{Vowel inventory}
  \label{tab:vowels}
\end{table}

The mid vowels \phomtext{e o} can be lowered \phontext{ɛ ɔ} in closed syllables.

Although \langname{} lacks diphthongs, any combination of monophthong and glide \phomtext{j w}
can function as the rime of a syllable, except \orthtext{iy}. \langname's syllable structure
also permits vowel hiatus, except vowel sequences 3 or longer containing two consecutive identical vowels.

\section{Syllable structure}
\langname~permits a maximum syllable structure of (C)V(C), where C is any consonant,
and V any monophthong. The syllable-initial sequence \orthtext{yi}---which resolves to \orthtext{i}---is prohibited, as is
the intervocalic sequence \orthtext{rr}, which resolves to \orthtext{r}.

\section{Assimilation patterns}
In a plosive--nasal sequence (between syllables), the plosive and nasal metathesise, and
the nasal assimilates to the place of articulation of the plosive. \phomtext{pak.ma} becomes
\orthtext{pangka}, for instance.

In a voiced plosive--voiceless plosive sequence, the voiced plosive devoices. \phomtext{pab.ka pad.ka}
become \orthtext{papka patka}, for instance.

\section{Stress and prosody}
\langname{} has regular penultimate stress. The stressed syllbale is often accompanied
by a higher pitch, and/or a lengthening of the vowel.