\chapter{Nouns}
\label{ch:nouns}

\section{Noun cases}
Nouns which become the \detail{focus} of the verb they are linked to do
not receive a case marking. The remaining arguments come with particles that indicate their role.

The \detail{oblique} and \detail{non-oblique case} mark the direct and indirect object
of a mono-/ditransitive verb. The \detail{genitive} marks the agent in a clause whose
verb is non-\AV. Just like their respective verb voices, the \detail{instrumental} and \detail{locative} mark the notions of `with'
and `at' respectively.
\begin{table}[htpb]
	\begin{tabular}{r l}
		\toprule
		\OBL  & \rom{aw}  \\
		\NOBL & \rom{ri}  \\
		\GEN  & \rom{ya}  \\
		\LOC  & \rom{ong} \\
		\INST & \rom{pa}  \\
		\bottomrule
	\end{tabular}
	\caption{Noun case markers}
	\label{tab:cases}
\end{table}

\section{Number}
\langname~does not explicitly mark for number. Full reduplication expresses
the \detail{collective noun} \COL, or the idea of the noun as opposed to an instance of it.

\section{Pronouns}
\langname~has a nice set of personal pronouns. When not the actor in a verb,
their full versions are used. When they act as the possessor of another noun, they
cliticise onto the possessee.
\begin{table}[htpb]
	\begin{tabular}{r l l}
		\toprule
		                & full          & \POSS        \\
		\midrule
		\FIRST\SG       & \rom{tsio}    & \rom{-o}     \\
		\FIRST\PL.\INCL & \rom{tsaen}   & \rom{-yen}   \\
		\FIRST\PL.\EXCL & \rom{tsoyang} & \rom{-yang}  \\
		\SECOND\SG      & \rom{nao}     & \rom{-ao}    \\
		\SECOND\PL      & \rom{naen}    & \rom{-(a)en} \\
		\THIRD\SG       & \rom{kew}     & \rom{-kyo}   \\
		\THIRD\PL       & \rom{kawang}  & \rom{-kwang} \\
		\bottomrule
	\end{tabular}
	\caption{Personal pronouns}
	\label{tab:personal_pronouns}
\end{table}

The possessor pronoun clitics can be compounded.
\begin{examples}
	\ex
	\label{ex:my_dads_dogs_food}
	\script andoo yebakyo arenganakyo
	\bits ando =o yeba =kyo are- nga -nak =kyo
	\gloss father 1SG.POSS dog 3SG.POSS NMZ eat PV 3SG.POSS
	\tr my father's dog's food
\end{examples}

\langname~also has a handy set of optional \detail{demonstratives} that don't conjugate for number.
They indicate definiteness.
\begin{table}[htpb]
	\begin{tabular}{r l}
		\toprule
		\DEM.\PROX & \rom{ea}   \\
		\DEM.\MED  & \rom{eho}  \\
		\DEM.\DIST & \rom{ekya} \\
		\bottomrule
	\end{tabular}
	\caption{Demonstratives}
	\label{tab:demonstratives}
\end{table}

\begin{examples}
	\ex
	\label{ex:spoilt_cheese}
	\script paonayoin ea adoyen
	\bits po- nayo -in ea adow =yen
	\gloss PFV spoil AV DEM.PROX cheese 1PL.INCL.POSS
	\tr This cheese (which is ours) has spoiled.
\end{examples}

\section{Verbal derivations}
Verbs with voice markings can be derived into nouns whose roles are
in some way related to the original verb's meaning. \INST~is not as productive
here.
\begin{examples}
	\ex
	\label{ex:eat_food}
	\script ngain arengain aw arenganak
	\bits nga -in are- nga -in aw are- nga -nak
	\gloss eat AV NMZ eat AV OBL NMZ eat PV
	\tr An eater eats the eaten.
	\tr Someone eats food.
\end{examples}

\begin{examples}
	\ex
	\label{ex:learn_school}
	\script tsidoraong aredoraong aw aredoranak
	\bits tsi= dora -ong are- dora -ong aw are- dora -nak
	\gloss 1SG.NOM learn LV NMZ learn LV OBL NMZ learn PV
	\tr A learning-place is learnt learnings in by me.
	\tr I study lessons at school.
\end{examples}