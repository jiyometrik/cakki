% REWRIITNG
\chapter{Nouns}
\label{ch:nouns}

\section{Noun cases}
Nouns which become the \detail{focus} of the verb they are linked to do
not receive a case marking. The remaining arguments come with particles that indicate their role.

The \detail{oblique} and \detail{indirect oblique} mark the direct and indirect object
of a mono-/ditransitive verb. The \detail{genitive} marks the agent in a clause whose
verb is non-\AV.
\begin{table}[htpb]
	\begin{tabular}{r l}
		\toprule
		\DIR  & \rom{wan} \\
		\OBL  & \rom{ri}  \\
		\NOBL & \rom{ka}  \\
		\LOC  & \rom{ong} \\
		\bottomrule
	\end{tabular}
	\caption{Noun case markers}
	\label{tab:cases}
\end{table}

In clauses involving verbs---as in \cref{ch:verbs}---the case markers attached
to nouns vary with the \detail{voice} of the verb.
\begin{table}[htpb]
	\begin{tabular}{r c c c c}
		\toprule
		voice & agent & direct patient & indirect patient & location \\
		\midrule
		\AV   & \DIR  & \OBL           & \NOBL            & \LOC     \\
		\PV   & \OBL  & \DIR           & \NOBL            & \LOC     \\
		\LV   & \OBL  & \OBL           & \NOBL            & \DIR     \\
		\IV   & \OBL  & \OBL           & \DIR             & \LOC     \\
		\bottomrule
	\end{tabular}
	\caption{Case-marking patterns in verb voices}
	\label{tab:voicepatterns}
\end{table}

\section{Number}
\langname~does not explicitly mark for number.
\begin{itemize}
	\item CV reduplication indicates the \detail{associative} (`...and such').
	\item Full reduplication indicates the \detail{collective} number.
\end{itemize}

\section{Pronouns}
\langname~has a nice set of personal pronouns. When used as an \rom{argument} in a clause, they cliticise to the ends of case markers.
When they act as the \rom{possessor} of another noun, they cliticise to the start the possessee.
\begin{table}[htpb]
	\begin{tabular}{r l l l}
		\toprule
		             & full         & argument    & \POS       \\
		\midrule
		\FIRST\S     & \rom{cio}    & \rom{-yo}   & \rom{ci-}  \\
		\FIRST\P.\IN & \rom{caen}   & \rom{-yen}  & \rom{cay-} \\
		\FIRST\P.\EX & \rom{cosang} & \rom{-yang} & \rom{co-}  \\
		\SECOND\S    & \rom{nao}    & \rom{-no}   & \rom{na-}  \\
		\SECOND\P    & \rom{naen}   & \rom{-nen}  & \rom{ney-} \\
		\THIRD\S     & \rom{kew}    & \rom{-ew}   & \rom{ke-}  \\
		\THIRD\P     & \rom{kawang} & \rom{-ang}  & \rom{kaw-} \\
		\bottomrule
	\end{tabular}
	\caption{Personal pronouns}
	\label{tab:personal_pronouns}
\end{table}

The possessor pronoun clitics can be compounded.
\begin{examples}
	\ex
	\label{ex:my_dads_dogs_ball}
	\script ciando keyeba kepoa
	\bits ci= ando ke= yeba ke= poa
	\gloss 1S.POS father 3S.POS dog 3S.POS ball
	\tr my father's dog's ball
\end{examples}

For \detail{inalienably possessed} nouns (e.g. kin, body parts), the
pronominal possessor clitic isn't necessary.
\begin{examples}
	\ex
	\label{ex:my_dads_dogs_nose}
	\script ciando keyeba iwco
	\bits ci= ando ke= yeba iwco
	\gloss 1S.POS father 3S.POS dog nose
	\tr my father's dog's nose
\end{examples}

\langname~also has a handy set of optional \detail{demonstratives} that don't conjugate for number.
They indicate nouns---they can be used as standalone pronouns or together with the linker (\LK) \rom{i} for definiteness.
\begin{table}[htpb]
	\begin{tabular}{r l}
		\toprule
		\PRX & \rom{ea}   \\
		\MED & \rom{eno}  \\
		\DST & \rom{yake} \\
		\bottomrule
	\end{tabular}
	\caption{Demonstratives}
	\label{tab:demonstratives}
\end{table}

\section{Noun phrases}
Noun phrases are predominantly \detail{head-final} --- the \detail{head noun}
of a noun phrase comes last. Each constituent is linked with \LK, or the same voice
marker used to mark the head noun.

To equate two nouns, they are juxtaposed.
\begin{example}
	\label{ex:woman_teacher}
	\script haha ea i pecim \\
	\bits haha ea i pecim \\
	\gloss woman PRX LK teacher \\
	\tr The teacher is a woman.
\end{example}
And to indicate the locative, the two nouns are also juxtaposed.
\begin{example}
	\label{ex:not_at_home}
	\script ihe ong sikaw yake i hea \\
	\bits ihe ong sikaw yake i hea \\
	\gloss NEG LOC hut DST LK boy \\
	\tr The boy isn't in the hut.
\end{example}
As mentioned in \cref{ch:verbs}, adjectives function like verbs---they take a voice
marker.
\begin{examples}
	\ex
	\label{ex:spoilt_cheese}
	\script ponayo porakso cayadowadow \\
	\bits po- nayo po- rakso cay= adow^ adow \\
	\gloss PFV be.bad PFV be.sour 1P.IN.POS COL cheese \\
	\tr All our cheese has spoilt and soured.

	\ex
	\label{ex:brain_rot}
	\script pokeaknak wan napido \\
	\bits po- keak -nak wan na= pido \\
	\gloss PFV rot PV DIR 2S.POS head \\
	\tr Your brain is rotten.
\end{examples}

\section{Subordinate clauses}
\detail{Subordinate clauses} are linked to their \detail{head noun} via \LK~too.
For subordinate clauses that denote an equality (to another noun), a description (with a stative verb), or a location, this is easy enough.
\begin{examples}
	\ex
	\label{ex:can_beat}
	\script angoy nginepnak rino wan yani i hea \\
	\bits angoy nginep -nak ri =no wan yani i hea \\
	\gloss POT be.above PV OBL 2S DIR be.weak LK boy \\
	\tr You could defeat the weak boy.

	\ex
	\label{ex:dont_beat_kaba}
	\script isop dasohnak wan Kaba i ea i ciyeba \\
	\bits isop dasoh -nak wan Kaba i ea i ci= yeba \\
	\gloss PRH beat PV DIR PN LK PRX LK 1S.POS dog \\
	\tr Don't beat Kaba, my dog!

	\ex
	\label{ex:from_bayma}
	\script yosroin ri kekorikori wan hay Bayma i omano \\
	\bits yosro -in ri ke= kori^ kori wan hay Bayma i omano \\
	\gloss rule AV OBL 3S.POS COL soldier DIR from PN LK chief \\
	\tr The chief from Baima ruled his army.
\end{examples}
More complex subordinate clauses are formed the same way.
\begin{examples}
	\ex
	\label{ex:there_is_his_friend}
	\script eno ngesinepin riew i mik \\
	\bits eno nginep <es> -in ri =ew i mik \\
	\gloss MED be.above IRR AV OBL 3S LK friend \\
	\tr There's the friend who'd lift him up.

	\ex
	\label{ex:there_is_his_friends}
	\script eno ngesinepnak riew i mik \\
	\bits eno nginep <es> -nak ri =ew i mik \\
	\gloss MED be.above IRR PV OBL 3S LK friend \\
	\tr There's the friend he'll lift up.
\end{examples}

\section{Classifiers}
When counting nouns, \detail{classifiers} are used to establish the type of noun.
There are classifiers for counting individual nouns---in particular, \CL.gust is used for counting events too.
\begin{table}[htpb]
	\begin{tabular}{r l}
		\toprule
		\CL.human  & \rom{cay}  \\
		\CL.mammal & \rom{kiw}  \\
		\CL.fish   & \rom{pok}  \\
		\CL.house  & \rom{inte} \\
		\CL.heavy  & \rom{pang} \\
		\CL.disc   & \rom{tie}  \\
		\CL.stick  & \rom{meo}  \\
		\CL.slice  & \rom{coah} \\
		\CL.pill   & \rom{ibo}  \\
		\CL.gust   & \rom{wara} \\
		\bottomrule
	\end{tabular}
	\caption{Demonstratives (individual)}
	\label{tab:demonstratives_ind}
\end{table}
\begin{examples}
	\ex
	\label{ex:coughed_once}
	\script hiwara pokohko yo \\
	\bits hi= wara po- kohko ∅ =yo \\
	\gloss one CL.gust PFV cough DIR 1S \\
	\tr I coughed one time.

	\ex
	\label{ex:three_eggs}
	\script beyranak wan meybo i boi \\
	\bits beyra -nak wan mey= bo i boi \\
	\gloss buy PV DIR three CL.pill LK egg \\
	\tr Three eggs are sold.
\end{examples}
% kayam/kay- - 0
% hia/hi- - 1
% bane/ba- - 2
% meyo/mey- - 3
% deyte/dey- - 4
% sitap/si- - 5
% roa/ro- - 6