\chapter{Verbs}
\label{ch:verbs}

\section{Voice}
\langname~features \detail{symmetrical voice}, where the role of the primary argument
(or the argument \detail{in focus}) is encoded onto the verb itself. There are four voices: \detail{actor voice} (\AV), \detail{patient voice} (\PV), \detail{locative voice} (\LV)
and \detail{instrumental voice} (\IV).

\begin{table}[htpb]
	\begin{tabular}{r l}
		\toprule
		\AV & \rom{-in}  \\
		\PV & \rom{-nak} \\
		\LV & \rom{-ong} \\
		\IV & \rom{-pay} \\
		\bottomrule
	\end{tabular}
	\caption{Voice affixes}
	\label{tab:voices}
\end{table}

\subsection{Actor voice}
\AV~brings into focus the agent of the action or state. In verbs that
take two arguments---an \detail{agent} and \detail{direct object}---the agent
is the focus.
\begin{examples}
	\ex
	\label{ex:actor_voice}
	\script ngain wan yeko ri mosea \\
	\bits nga -in wan yeko ri mosea \\
	\gloss eat AV DIR child OBL crab \\
	\tr A child eats crab.
\end{examples}
For \detail{intransitive} verbs that only take one argument (again the agent), \AV~is used
to indicate the causative form of the verb.
\begin{examples}
	\ex
	\label{ex:noncaus}
	\script sapek cai \\
	\bits sapek cai \\
	\gloss be.angry man \\
	\tr That man is angry.
\end{examples}
\begin{examples}
	\ex
	\label{ex:caus}
	\script sapekin wan cai ri mosea \\
	\bits sapek -in wan cai ri mosea \\
	\gloss be.angry AV DIR man OBL crab \\
	\tr That man angered the crab.
\end{examples}

\subsection{Patient voice}
\PV~brings into focus the \detail{direct object}---the \detail{patient}---of
the action.
\begin{examples}
	\ex
	\label{ex:pigcaught}
	\script koparnak wan bao ri boyan \\
	\bits kopar -nak wan bao ri boyan \\
	\gloss catch PV DIR pig OBL girl \\
	\tr The pig is caught by the girl.
\end{examples}

\PV~may also decrease the valency of a transitive verb, so that
the patient is the only required argument.
\begin{examples}
	\ex
	\label{ex:kidinjured}
	\script ocitnak hea \\
	\bits ocit -nak hea \\
	\gloss injure PV boy \\
	\tr The boy is hurt.
\end{examples}

\subsection{Locative voice}
\LV~brings into focus the location where an action is performed or experienced.
The agent and direct patient are both marked in the \detail{oblique} case here,
so the agent must precede the direct patient.
\begin{examples}
	\ex
	\label{ex:carshop}
	\script beyraong ri kori wan makesen ri soam \\
	\bits beyra -ong ri kori wan makesen ri soam \\
	\gloss buy LV DIR shop OBL car OBL soldier  \\
	\tr The shop is where the soldier buys cars. \\
\end{examples}
\LV~may also bring into focus the argument for whom an action is performed.
\begin{examples}
	\ex
	\label{ex:slaughter}
	\script hoakong ri eno i wewin wan inan i toan \\
	\bits hoak -ong ri eno i wewin wan inan i toan \\
	\gloss kill LV OBL MED LK goat DIR god LK sun \\
	\tr For the sun god is the goat slaughtered.
\end{examples}
\begin{examples}
	\ex
	\label{ex:givecake}
	\script angoy ondaong wanno ri ciayon  \\
	\bits angoy onda -ong wan =no ri ci= ayon \\
	\gloss POT give LV DIR 2S OBL 1S.POS soul \\
	\tr I can give you my soul.
\end{examples}
\LV~in general emphasises the secondary patient of an action.
\begin{examples}
	\ex
	\label{ex:borrowbooks}
	\script bidanong ri hea wan boyan ri caik \\
	\bits bidan -ong ri hea wan boyan ri caik \\
	\gloss borrow LV OBL boy DIR girl OBL book \\
	\tr A boy borrows a book from a girl.
\end{examples}
With only two arguments (the actor and direct object), \LV~may also take
on a \detail{partitive} role.
\begin{examples}
	\ex
	\label{ex:cookmeat}
	\script kohanong wan obaw ri ciamha \\
	\bits kohan -ong wan obaw ri ci= amha \\
	\gloss cook LV DIR meat OBL 1S.POS mother \\
	\tr At meat does my mother cook.
	\tr My mother cooks some meat.
\end{examples}


\subsection{Instrumental voice}
\IV~primarily indicates the \detail{means} with which an action is performed.
The agent and patient are similarly both marked with the \detail{oblique},
so the agent precedes the patient.
\begin{examples}
	\ex
	\label{ex:digsoil}
	\script bahsipay ri hehea wan copip ri ingga \\
	\bits bahsi -pay ri he^ hea wan copip ri ingga \\
	\gloss dig IV OBL ASS boy DIR hoe OBL soil \\
	\tr Hoes are dug soil with by boys.
\end{examples}
\IV~can also indicate an agent alongside which an action is performed (i.e. the comitative).
\begin{examples}
	\ex
	\label{ex:singing}
	\script odepay ri boyan wan hea \\
	\bits ode -pay ri boyan wan hea \\
	\gloss sing IV OBL girl DIR boy \\
	\tr The girl sings with the boy.
\end{examples}

\section{Tense}
\detail{Stative verbs}---verbs that describe a state of being--and \detail{dynamic verbs}---those
that describe an action, receive an identical set of tense affixes.

There are three broad `tenses'. The \detail{perfective} describes completed actions,
the \detail{imperfective} describes actions that have yet to take place, and
the \detail{progressive} describes actions that are in the process of completion.
\begin{table}[htpb]
	\begin{tabular}{r l}
		\toprule
		\PFV  & \rom{po-}                         \\
		\NPFV & \rom{∅}, \rom{nge-}               \\
		\PRG  & first syllable/full reduplication \\
		\bottomrule
	\end{tabular}
	\caption{Tense affixes}
	\label{tab:tenses}
\end{table}

\PFV~implies an action's full completion; \NPFV~sometimes implies that the verb is a general truth, or that it is known to happen.
Full reduplication in \PRG~implies that the action is tedious or wearisome to complete,
or that it is nowhere close to ending.

\begin{examples}
	\ex
	\label{ex:isoldsomerice}
	\script poyamiong wan sisici riyo \\
	\bits po- yami -ong wan si^ sici ri =yo \\
	\gloss PFV sell LV DIR ASS rice OBL 1S \\
	\tr I sold some rice. \\

	\ex
	\label{ex:isellsomericeeveryday}
	\script ong noy i soay ngeyamiong wan sisici riyo \\
	\bits ong noy i soay nge- yami -ong wan si^ sici ri =yo \\
	\gloss LOC every LK day NPFV sell LV DIR ASS rice OBL 1S \\
	\tr Every day I (regularly) sell rice. \\
\end{examples}

\section{Aspect and mood}
Ditto, stative and dynamic verbs receive identical aspect and mood markers.
\begin{table}[htpb]
	\begin{tabular}{r l}
		\toprule
		\IRR & \rom{⟨es⟩} \\
		\IMP & \rom{da-}  \\
		\PRH & \rom{isop} \\
		\INT & \rom{-we}  \\
		\bottomrule
	\end{tabular}
	\caption{Aspect and mood affixes}
\end{table}