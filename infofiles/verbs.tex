\chapter{Verbs}
\label{ch:verbs}

\section{Voice}
Monotransitive and ditransitive verbs take one of the following suffixes to
indicate the role of the argument \detail{in focus}. Avalent verbs receive no such treatment.
\begin{table}[htpb]
	\begin{tabular}{r l}
		\toprule
		\AV & \rom{-in}  \\
		\PV & \rom{-nak} \\
		\LV & \rom{-ong} \\
		\IV & \rom{-paw} \\
		\bottomrule
	\end{tabular}
	\caption{Voice affixes}
	\label{tab:voices}
\end{table}

\section{Person}
Verbs also conjugate for person if the \detail{actor} in the verb can be expressed
with a personal pronoun. The personal pronouns cliticise onto the verb.
\begin{table}[htpb]
	\begin{tabular}{r l}
		\toprule
		\FIRST\SG       & \rom{tsi-}  \\
		\FIRST\PL.\INCL & \rom{tsay-} \\
		\FIRST\PL.\EXCL & \rom{tsod-} \\
		\SECOND\SG      & \rom{na-}   \\
		\SECOND\PL      & \rom{ney-}  \\
		\THIRD\SG       & \rom{∅}     \\
		\THIRD\PL       & \rom{kaw-}  \\
		\bottomrule
	\end{tabular}
	\caption{Verbal pronoun clitics}
	\label{tab:pronoun_clitic}
\end{table}

\section{Tense}
\detail{Stative verbs}---verbs that describe a state of being--and \detail{dynamic verbs}---those
that describe an action, receive identical tense affixes.

There are three broad `tenses'. The \detail{perfective} describes completed actions,
the \detail{imperfective} describes actions that have yet to take place, and
the \detail{progressive} describes actions that are in the process of completion.
\begin{table}[htpb]
	\begin{tabular}{r l}
		\toprule
		\PFV  & \rom{po-}                         \\
		\NPFV & \rom{∅}, \rom{nge-}               \\
		\PROG & first syllable/full reduplication \\
		\bottomrule
	\end{tabular}
	\caption{Tense affixes}
	\label{tab:tenses}
\end{table}
Full reduplication in \PROG~implies that the action is tedious or wearisome to complete,
or that it is nowhere close to ending.

\section{Aspect and mood}
Ditto, stative and dynamic verbs receive identical aspect and mood markers.
\begin{table}[htpb]
	\begin{tabular}{r l}
		\toprule
		\HAB & \rom{tang-} \\
		\IRR & \rom{sin-}  \\
		\bottomrule
	\end{tabular}
	\caption{Aspect and mood affixes}
\end{table}
In addition, the standalone particle \rom{hew} is used in conjunction with
any of the verbal affixes to indicate the interrogative.

So far, no irregular verbs have cropped up.