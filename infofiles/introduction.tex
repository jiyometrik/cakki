\chapter{Introduction}
\langname~is an attempt at a naturalistic constructed language (or \detail{conlang}, for short).
It aims to pass off as a language that could feasibly exist in our world. This sort of
language has fascinated me for very long: how do you make something constructed \detail{believable}?

I've been lurking within the consturcted language sector, and more broadly, the linguistics
sector of the Internet for about a year. In that time, I tried my hand at other naturalistic \detail{conlangs},
none of which I found particularly satisfactory.

In the real world, \langname{} is heavily inspired by the Formosan languages of Taiwan,
particularly Paiwan (\cite{chang_paiwan}) and Saaroa (\cite{pan_saaroa}).
Their phonological aesthetics and intriguing grammars heavily inspired this conlang.
Admittedly, the author's amateurish dabbling in Austronesian alignment isn't reflected in this language,
mostly because I've yet to grasp \textit{how} it evolved and how to emulate such a system.

Speaking of, the linguistic features in this language are not nearly as
complex or unique as those of \textit{higher} conlangs, but this conlang
was never meant to be creative or earth-shattering in its conception. \langname{}
is my personal creation; it is something for me to take pride in.

Let \langname~be the very first time I stick through and present something I'm personally proud of.
Let \langname~be my \detail{magnum opus}, if you will.\sidenote{Insert some cool phrase in \langname~here!}
