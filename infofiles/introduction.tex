\chapter{Introduction}
\section{Meta history}
The \langname{} language was conceived of as a conlang and nothing else. No worldbuilding,
no fancy artistic philosophy, no obvious self-expressive elements. I designed \langname{}
as a way to pass the time, among so many other life stuff. The best way to think about
\langname{} is to consider it a \detail{speedlang}, with a timeframe so stretched out a more professional
conlanger could have completed a real conlang already. But still we persist.

\langname{} is my sanity saver and time waster. For it, I thank conlanging and all my
inspirations---Biblaridion, u/mareck, et al.---for existing.

\section{Conventions}
Per convention, phonemic and phonemic realisations of words are in \phomtext{forward slashes}
and \phontext{square brackets}. Romanisations are in \orthtext{angle brackets}, lexemes are in
\rom{colour}, and new linguistic terms are in \detail{italics}.