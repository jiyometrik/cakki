\chapter{Nominals}
\label{cha:nominals}

This chapter discusses the \langname{} noun. The most essential---the pronouns---are
introduced in \cref{sec:pronouns}, following which grammatical number is discussed (\cref{sec:noun_number}).
A case system (\cref{sec:case_system}) is then presented.
Notably, \langname{} does not contain noun classes which are to be treated differently syntactically.

\section{Pronouns and determiners}
\label{sec:pronouns}

\subsection{Personal pronouns}
The personal pronouns differentiate three persons (first, second and third) and two grammatical numbers (singular and plural).
They do not distinguish grammatical gender nor animacy.
Notably, the first person plural also distinguishes clusivity: there are separate roots for the first
person plural inclusive and first person plural exclusive.

The case markers for the three nuclear cases remain intact on all seven
pronoun roots. Where the number of syllables becomes unwieldy, the first syllables
of the pronouns are truncated.

As seen later, pronouns become cliticised when verbs are introduced---it is mandatory for a verb
to denote its agent and direct patient. \Cref{tab:personal_pronouns} lists two versions
of each pronoun: the former being the pronoun root (in reality), and the
latter being the cliticised form to be attached to verbs in a clause.\sidenote{The third person singular is treated as the default person, so it does not need to adopt a clitic form.}
The clitic suffixes in the absolutive case are often subsumed into the dative case for certain verbs
that involve the transfer of objects from one noun to another.

\begin{table}[htbp]
  \centering
  \begin{tabular}{c c c c c}
    \toprule
    & \ABS{} & \ERG{} & \DAT{} \\
    \midrule
    \FIRST\SG{} & \rom{sinaw}, \rom{-naw} & \rom{le-sinaw}, \rom{si-} & \rom{nu-sinaw} \\
    \FIRST\PL.\INCL{} & \rom{ŋenakay}, \rom{-kay} & \rom{le-nakay}, \rom{ŋena-} & \rom{nu-ŋakay} \\
    \FIRST\PL.\EXCL{} & \rom{ŋurumay}, \rom{-may} & \rom{le-rumay}, \rom{ŋuru-} & \rom{nu-rumay} \\
    \SECOND\SG{} & \rom{ketaw}, \rom{-taw} & \rom{le-ketaw}, \rom{ke-} & \rom{nu-ketaw} \\
    \SECOND\PL{} & \rom{kacelu}, \rom{-lu} & \rom{le-celu}, \rom{kace-} & \rom{nu-celu} \\
    \THIRD\SG{} & \rom{mera} & \rom{le-mera} & \rom{nu-mera} \\
    \THIRD\PL{} & \rom{warhal}, \rom{-hal} & \rom{le-warhal}, \rom{war-} & \rom{nu-warhal} \\
    \bottomrule
  \end{tabular}
  \caption{Personal pronouns}
  \label{tab:personal_pronouns}
\end{table}

Since the personal pronouns are always cliticised onto verbs, they are very often
omitted in complete sentences, as in this extreme \cref{ex:we_kill_them}.
\begin{example}
  \label{ex:we_kill_them}
  \script Siwahashal maylecelu.
  \gloss
  si= & 1SG.ERG \\
  wahas & kill \\
  =hal & 3PL.DAT \\
  may- & INS \\
  le- & ABS \\
  celu & 2PL \\
  \tr I kill them alongside you (all).
\end{example}

The pronoun clitics are also used for denoting possession of a noun by another.
The grammatical case the `possessor' clitics adopt agrees with that of the `possession', and
the `possessor' noun follows the `possession'. Where the `possessor' noun is a pronoun unto itself,
the case marker can be omitted, since the `possessor' pronoun clitic already reveals its case.

Compare \cref{ex:i_hate_their_dog,ex:they_hate_my_dog}: notice how the pronoun clitics attached to the dog \rom{pariw}
change to denote the possessor, according to the dog's role in the sentence.

\begin{example}
  \label{ex:i_hate_their_dog}
  \script Siyahin pariwhal.
  \gloss
  si= & 1SG.ERG \\
  yahin & hate \\
  =∅ & 3SG.ABS \\
  pariw & dog \\
  =hal & 3PL.ABS \\
  \tr I hate their dog.
\end{example}

\begin{example}
  \label{ex:they_hate_my_dog}
  \script Waryahin pariwnaw.
  \gloss
  war= & 3PL.ERG \\
  yahin & hate \\
  =∅ & 3SG.ABS \\
  pariw & dog \\
  =naw & 1SG.ABS \\
  \tr They hate my dog.
\end{example}

Compare also \cref{ex:i_see_pamats_eye,ex:i_see_his_eye}. In both examples,
the `possessor' takes on the absolutive case while the `possession'---the eye---is
in the dative case. Appending \rom{mera} in \cref{ex:i_see_his_eye} is essential here
to clarify that the eye is being possessed, rather than being simply an eye.

\begin{example}
  \label{ex:i_see_pamats_eye}
  \script Si'umpa nuŋe' Pamat.
  \gloss
  si= & 1SG.ERG \\
  'umpa & see \\
  =∅ & 3SG.DAT \\
  nu- & DAT \\
  ŋe' & eye \\
  ∅= & ABS \\
  Pamat & Pamat \\
  \tr I see Pamat's eye.
\end{example}

\begin{example}
  \label{ex:i_see_his_eye}
  \script Si'umpa nuŋe' mera.
  \gloss
  si= & 1SG.ERG \\
  'umpa & see \\
  =∅ & 3SG.DAT \\
  nu- & DAT \\
  ŋe' & eye \\
  mera & 3SG.ABS \\
  \tr I see his/her eye.
\end{example}

\subsection{Demonstrative pronouns---or maybe not?}
\langname{} lacks demonstrative pronouns (the equivalents of \detail{this} and \detail{that} in English).
Since the third person pronouns do not distinguish animacy, they are also used when referring
to objects. Instead, all nouns are treated as inherently indefinite. To mark definiteness,
the prefix \rom{pel-} is used. Observe the contrast between \cref{ex:slit_bamboo,ex:slit_the_bamboo}.

\begin{example}
  \label{ex:slit_bamboo}
  \script Teŋahal miminahaw.
  \gloss
  ∅= & 3SG.ERG \\
  teŋa & slit \\
  =hal & 3PL.ABS \\
  ∅- & ABS \\
  mi^ & COL \\
  minahaw & bamboo \\
  \tr He/she slits bamboo (in general).
\end{example}

\begin{example}
  \label{ex:slit_the_bamboo}
  \script Teŋahal pelmiminahaw.
  \gloss
  ∅= & 3SG.ERG \\
  teŋa & slit \\
  =hal & 3PL.ABS \\
  ∅- & ABS \\
  pel- & DEF \\
  mi^ & COL \\
  minahaw & bamboo \\
  \tr He/she slits the bamboo.
\end{example}

Grammatical number is already established by the pronoun root itself, so
equivalents of English \detail{these} and \detail{those} aren't required.
Interestingly, the definite prefix can also be used when introducing a person
by name, as in \cref{ex:i_am_hayat}.
\begin{example}
  \label{ex:i_am_hayat}
  \script Pelhayat'urah sinaw.
  \gloss
  pel- & DEF \\
  Hayat'urah & Khayatura \\
  sinaw & 1SG.ABS \\
  \tr I am Khayatura.
\end{example}

This `introductory' function of the definite prefix also extends to other nouns.
Broadly, it can be employed to draw attention to a noun, when, for instance, introducing
it for the first time.
% TODO add ex here

\section{Grammatical number}
\label{sec:noun_number}

\langname{} nouns do not inflect for grammatical number. In actuality,
nouns are free to be interpreted with any number. Contrast \cref{ex:sheep_happy,ex:sheeps_happy}:
the only hint at whether the \detail{sheep} is an individual or a number of sheep is
the third person marker on the verb, rather than on the noun.

\begin{example}
  \label{ex:sheep_happy}
  \script Yayamu pelhurma.
  \gloss
  ya^ & PROG \\
  yamu & happy \\
  =∅ & 3SG.ABS \\
  pel- & DEF \\
  hurma & sheep \\
  \tr The sheep is happy.
\end{example}

\begin{example}
  \label{ex:sheeps_happy}
  \script Yayamuhal pelhurma.
  \gloss
  ya^ & PROG \\
  yamu & happy \\
  =hal & 3PL.ABS \\
  pel- & DEF \\
  hurma & sheep \\
  \tr The sheep are happy.
\end{example}

Despite the lack of a distinction between singular and plural nouns (marked on the noun), it would be remiss to neglect
the role \textbf{reduplication} plays. Reduplication is used to express a noun in the collective (\COL{});
instead of referring to the noun as an individual or in a number, the noun can be semantically expressed
as a general concept. Contrast again \cref{ex:sheep_happy,ex:sheeps_happy} with \cref{ex:the_sheeps_happy}.
\begin{example}
  \label{ex:the_sheeps_happy}
  \script Yayamuhal pelhurhurma.
  \gloss
  ya^ & PROG \\
  yamu & happy \\
  =hal & 3PL.ABS \\
  pel- & DEF \\
  hu^ & COL \\
  hurma & sheep \\
  \tr The sheep (in general, as a concept) are happy.
\end{example}

To express the collective number, the first CV pair of the noun root is reduplicated.
This reduplication can occur with any noun, as in \cref{ex:dog_singular,ex:dog_collective}.
\begin{examples}
  \ex\label{ex:dog_singular}
  \script pariw
  \gloss pariw & dog \\
  \tr a dog/some dogs

  \ex\label{ex:dog_collective}
  \script papariw
  \gloss
  pa^ & COL \\
  pariw & dog \\
  \tr dogs (as a collective)
\end{examples}

\section{Case system}
\label{sec:case_system}

Generally, \langname{}'s case system is highly productive, containing
a variety of cases marking the roles of nouns in a sentence.

\subsection{Nuclear cases}

\langname{} relies on ergative--absolutive alignment, which means that the subject
of an intransitive statement is marked the same as the direct patient (or object) of a transitive
statement, with the \detail{absolutive case} (\ABS{}), marked with a zero morpheme. \Cref{ex:abs_intransitive} shows
the absolutive case on intransitive subjects.\sidenote{Note also that the absolutive noun is marked on the verb with a suffix!}

\begin{example}
  \label{ex:abs_intransitive}
  \script Tuŋakŋak Luwan.
  \gloss
  tu- & PST \\
  ŋak^ & PROG \\
  ŋak & whine \\
  =∅ & 3SG.ABS \\
  ∅- & ABS \\
  Luwan & Luan \\
  \tr Luan was whining.
  \not Luan whined.
\end{example}

For transitive statements involving an agent and a patient,
the patient adopts the absolutive case as before, and the
agent adopts the \detail{ergative case} (\ERG{}), marked with the prefix \rom{le-}. This is shown
in \cref{ex:erg_transitive}, where \rom{luwan} is the agent and \rom{'alik} is
the patient.\sidenote{Similarly, the ergative noun is marked on the verb with a prefix!}

\begin{example}
  \label{ex:erg_transitive}
  \script Tuyahin lepariw 'alik.
  \gloss
  tu- & PST \\
  ∅= & 3SG.ERG \\
  yahin & hate \\
  =∅ & 3SG.ABS \\
  le- & ERG \\
  pariw & dog \\
  ∅- & ABS \\
  'alik & daikon \\
  \tr A dog hated daikon.
  \not A dog \textit{was hating} daikon.
\end{example}

A third dative case (\DAT) exists in \langname{} to mark the indirect
patient of a transitive statement. The three parties in \cref{ex:erg_abs_dat} (agent,
direct patient and indirect patient) can then be crudely
identified as the object performing the action, the object being acted upon,
and the beneficiary of the action. Similarly, the dative case is marked with the prefix \rom{nu-}.

\begin{example}
  \label{ex:erg_abs_dat}
  \script Tu'alit leluwan nuhayat 'alik.
  \gloss
  tu- & PST \\
  ∅= & 3SG.ERG \\
  'alit & give \\
  =∅ & 3SG.ABS \\
  le- & ERG \\
  Luwan & Luan \\
  nu- & DAT \\
  Hayat & Khayat \\
  ∅- & ABS \\
  'alik & daikon \\
  \tr Luan gave daikon to Khayat.
  \not Luan gave \textit{the} daikon to Khayat.
\end{example}

The dative case can also be used when an action inherently involves a transfer of
an object from one noun to another noun, as in \cref{ex:feeding_chicken}.
\begin{example}
  \label{ex:feeding_chicken}
  \script Naŋkit nucarik.
  \gloss
  ∅= & 3SG.ERG \\
  naŋkit & feed \\
  =∅ & 3SG.DAT \\
  nu- & DAT \\
  carik & chicken \\
  \tr He/she feeds (a) chicken.
\end{example}

So far, the word order has remained fixed---a steady VS for intransitive statements (where V is the verb and S is the subject),
and a VA(I)P for transitive statements, where A is the agent, I is the indirect patient, and P is the direct patient.

To mark the \detail{topic} of a statement, the particle \rom{ya} is used (\TOP) and the topic is commonly
(but not always) shifted to the front of the sentence. Using the previous example
as a reference, \cref{ex:erg_focus,ex:abs_focus,ex:dat_focus} show this. If the topic already contains
a non-zero role marker (like \ERG{} or \DAT), the marker is retained.

\begin{examples}
  \ex\label{ex:erg_focus}
  \script Leluwan ya tu'alit nuhayat 'alik.
  \gloss
  le- & ERG \\
  Luwan & Luan \\
  ya & TOP \\
  tu- & PST \\
  ∅= & 3SG.ERG \\
  'alit & give \\
  =∅ & 3SG.ABS \\
  nu- & DAT \\
  Hayat & Khayat \\
  ∅- & ABS \\
  'alik & daikon \\
  \tr Regarding Luan, he gave daikon to Khayat.
\end{examples}

\begin{examples}
  \ex\label{ex:abs_focus}
  \script 'alik ya tu'alit leluwan nuhayat.
  \gloss
  ∅- & ABS \\
  'alik & daikon \\
  ya & TOP \\
  tu- & PST \\
  ∅= & 3SG.ERG \\
  'alit & give \\
  =∅ & 3SG.ABS \\
  le- & ERG \\
  Luwan & Luan \\
  nu- & DAT \\
  Hayat & Khayat \\
  \tr Regarding daikon, Luan gave it to Khayat.
\end{examples}

\begin{examples}
  \ex\label{ex:dat_focus}
  \script Nuhayat ya tu'alit leluwan 'alik.
  \gloss
  nu- & DAT \\
  Hayat & Khayat \\
  ya & TOP \\
  tu- & PST \\
  ∅= & 3SG.ERG \\
  'alit & give \\
  =∅ & 3SG.ABS \\
  le- & ERG \\
  Luwan & Luan \\
  ∅- & ABS \\
  'alik & daikon \\
  \tr Regarding daikon, Luan gave it to Khayat.
\end{examples}

\subsection{Other cases}
\langname{} has one case that mark the position of an object relative to another---the
locative (\LOC, \detail{at} an object). The locative may be used in conjunction
with any of the three nuclear cases mentioned above.

The locative case denotes an object at approximately the same location as another.
\Cref{ex:willow_walk,ex:rock_fight} show the locative case in action in intransitive
and transitive statements.

\begin{examples}
  \ex\label{ex:willow_walk}
  \script Pasinaw cinraraŋuy.
  \gloss
  pasi & walk \\
  =naw & 1SG.ABS \\
  cin- & LOC \\
  ra^ & COL \\
  raŋuy & willow \\
  \tr I walk in the willows.
\end{examples}

\begin{examples}
  \ex\label{ex:rock_fight}
  \script Simeluktaw cintentu.
  \gloss
  si= & 1SG.ERG \\
  meluk & fight \\
  =taw & 2SG.ABS \\
  cin- & LOC \\
  tentu & rock \\
  \tr I fight you at a rock.
\end{examples}

In addition, the instrumental case (\INS) is employed to denote an action
undergone \detail{using} an object, as seen in \cref{ex:kill_knife}. The instrumental also subsumes the comitative
case present in other languages---it can also denote an action undergone \detail{in conjunction with}
an object. Again, the instrumental case can be attached to another of the nuclear cases.

\begin{example}
  \label{ex:kill_knife}
  \script Siwahas (lesinaw) nupamat maymiraŋ.
  \gloss
  si= & 1SG.ERG \\
  wahas & kill \\
  =∅ & 3SG.DAT \\
  le- & ERG \\
  sinaw & 1SG \\
  nu- & DAT \\
  Pamat & Pamat \\
  may- & INS \\
  miraŋ & dagger \\
  \tr I kill Pamat with a dagger.
\end{example}

\begin{example}
  \label{ex:kill_chickens}
  \script Siwahas (lesinaw) maypamat nucarik.
  \gloss
  si= & 1SG.ERG \\
  wahas & kill \\
  =∅ & 3SG.DAT \\
  le- & ERG \\
  sinaw & 1SG \\
  may- & INS \\
  Pamat & Pamat \\
  nu- & DAT \\
  carik & chicken \\
  \tr I kill (a) chicken with Pamat.
\end{example}
