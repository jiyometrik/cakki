\chapter{Nominals}
\label{cha:nominals}

This chapter discusses the \langname{} noun. The most essential---the pronouns---are
introduced in \cref{sec:pronouns}, following which grammatical number is discussed (\cref{sec:noun_number}).
A case system (\cref{sec:case_system}) is then presented.
Notably, \langname{} does not contain noun classes which are to be treated differently syntactically.

\section{Pronominals}
\label{sec:pronouns}

\subsection{Personal pronouns}
The personal pronouns differentiate three persons (first, second and third) and two grammatical numbers (singular and plural).
They do not distinguish grammatical gender nor animacy.
Notably, the first person plural also distinguishes clusivity: there are separate roots for the first
person plural inclusive and first person plural exclusive.

Pronouns very frequently take on cliticised forms: on verbs
to mark their subject, and on other referents to mark their possessor.
\Cref{tab:personal_pronouns} lists three forms
of each pronoun: one, the standalone pronoun root;
two, cliticised forms in the \acrfull{nom}\sidenote{The third person singular is treated as the default person, so it does not need to adopt a clitic form in the \acrfull{nom}.};
and three, cliticised forms in the \acrfull{gen} to mark a pronoun's possession of another referent.
Pronominal clitics in other cases, like the \acrfull{acc} and \acrfull{dat}, do not exist.

\begin{table}[htbp]
  \centering
  \begin{tabular}{c c c c}
    \toprule
    & Standalone & Clitic (\NOM{}) & Clitic (\GEN{}) \\
    \midrule
    \FIRST\SG{} & \rom{seti} & \rom{-ti} & \rom{se-} \\
    \FIRST\PL.\INCL{} & \rom{tekay} & \rom{-kay} & \rom{tek-} \\
    \FIRST\PL.\EXCL{} & \rom{huruy} & \rom{-ruy} & \rom{hur-} \\
    \midrule
    \SECOND\SG{} & \rom{kawi} & \rom{-wi} & \rom{ka-} \\
    \SECOND\PL{} & \rom{kempul} & \rom{-pul} & \rom{kem-} \\
    \midrule
    \THIRD\SG{} & \rom{mera} & ∅ & \rom{me-} \\
    \THIRD\PL{} & \rom{halwat} & \rom{-wat} & \rom{hal-} \\
    \bottomrule
  \end{tabular}
  \caption{Personal pronouns}
  \label{tab:personal_pronouns}
\end{table}

Since a pronominal subject is always cliticised onto a verb, \langname{}
is a largely pro-drop language, as in the very simple \cref{ex:i_kill}.
The first person singular \rom{seti} is completely omitted from the sentence,
because the first person singular clitic \rom{-ti} is sufficient to indicate the subject.

\begin{example}
  \label{ex:i_kill}
  \gloss
  wahas & kill \\
  =ti & 1SG.NOM \\
  ta- & ACC \\
  mera & 3SG \\
  \tr I kill him/her.
\end{example}

Observe also the differences between clitics in the \acrfull{nom} and \acrfull{gen},
as in \cref{ex:you_feed_your_dog}, where the second person singular occur in both
the \acrfull{nom} and \acrfull{gen}.
\begin{example}
  \label{ex:you_feed_your_dog}
  \gloss
  kela- & CAUS \\
  lam & eat \\
  =\highlight{wi} & 2SG.NOM \\
  saŋ- & DAT \\
  \highlight{ka}= & 2SG.GEN \\
  pariw & dog \\
  \lit You cause your dog to eat.
  \tr You feed your dog.
\end{example}

It should be noted that the personal pronouns are not only used for \textit{persons}, but
for any `concrete' nouns, even inanimate objects, as in \cref{ex:sun_loves_me}, where
\rom{kuhis} (`sun') is the subject, yet it still takes on the third person singular
clitic.
\begin{example}
  \label{ex:sun_loves_me}
  \gloss
  murmuni & honorific.female \\
  kuhis & sun \\
  saŋ- & DAT \\
  seti & 1SG \\
  cica & love \\
  =\highlight{∅} & 3SG.NOM \\
  \lit Madam Sun loves me.
  \tr The sun loves me.
\end{example}

\subsection{Demonstrative pronouns}
In \langname{}, there are two pronouns in the \acrfull{dem}: the \acrfull{prox}
demonstrative \rom{ʔamuy} and the \acrfull{dist} demonstrative \rom{ʔaliŋ}.
On their own, they act similarly to the personal pronouns, as in \cref{ex:this_is_hilew_flower,ex:that_made_you_laugh}.

\begin{example}
  \label{ex:this_is_hilew_flower}
  \gloss
  ʔamuy & DEM.PROX \\
  ta- & ACC \\
  Hilew & Hilo \\
  me= & 3SG.GEN \\
  yukuha & flower \\
  \tr This is Hilo's flower.
\end{example}

\begin{example}
  \label{ex:that_made_you_laugh}
  \gloss
  ʔaliŋ & DEM.DIST \\
  saŋ- & DAT \\
  kempul & 2PL \\
  kela- & CAUS \\
  haʔik & laugh \\
  =∅ & 3SG.NOM \\
  \lit That caused you (pl.) to laugh.
  \tr That made you all laugh.
\end{example}

When following other referents, however, the demonstratives act as determiners
that refer to their respective referents, as in \cref{ex:hilos_flower,ex:sick_children}.

\begin{example}
  \label{ex:hilos_flower}
  \gloss
  Hilew & Hilo \\
  me= & 3SG.GEN \\
  yukuha & flower \\
  \highlight{ʔamuy} & DEM.PROX \\
  kuyani & yellow\\
  =∅ & 3SG.NOM \\
  \tr This flower of Hilo's is yellow.
\end{example}

\begin{example}
  \label{ex:sick_children}
  \gloss
  sarin^ & COL \\
  sarin & child \\
  \highlight{ʔaliŋ} & DEM.DIST \\
  puŋ- & PST \\
  (ʔ)utuk & sick \\
  =wat & 3PL.NOM \\
  \tr Those children (as a group) were sick.
\end{example}

This `pointer' function of the demonstratives is especially helpful
when combined with the personal pronouns, to create a pseudo-proximate--obviative
dichotomy. When the same personal pronoun occupies both the subject and the object
position in a clause (despite each personal pronoun referring to a different person),
the demonstratives help to distinguish them.

\Cref{ex:they_fight_them} has the third
person plural \rom{halwat} occupying both the subject and the object. With
the demonstrative \rom{ʔaliŋ}, however, the first third person plural (indicated by \rom{-wat} cliticised
onto the verb) can be assumed to be known by the speaker, or closer in distance to the speaker.
The second third person plural marked in the \acrfull{acc} can then be
assumed to be foreign to the speaker, further away from him/her, or
simply a new group of people not yet mentioned by the speaker previously.

\begin{example}
  \label{ex:they_fight_them}
  \gloss
  meluk & fight \\
  =wat & 3PL.NOM \\
  ta- & ACC \\
  halwat & 3PL \\
  \highlight{ʔaliŋ} & DEM.DIST \\
  \tr They (prox.) fight them (dist.).
\end{example}

Furthermore, the demonstratives can also be used to add emphasis to proper names
when they are introduced. In some English dialects, this behaviour has an equivalent---`\textit{that} John'.
This idiosyncrasy is illustrated in \cref{ex:luwan_sees_pamat}.

\begin{example}
  \label{ex:luwan_sees_pamat}
  \gloss
  Luwan & Luan \\
  \highlight{ʔamuy} & DEM.PROX \\
  ta- & ACC \\
  Pamat & Pamat \\
  nuk^ & PROG \\
  nuk & see \\
  =∅ & 3SG.NOM \\
  \lit This Luan is seeing Pamat.
  \tr \textit{Luan} is looking at Pamat.
\end{example}

\section{Grammatical number}
\label{sec:noun_number}

\langname{} doesn't grammatically mark the differences between \acrfull{sg} and \acrfull{pl}
nouns---this information is left up to context. However, a \textbf{\acrfull{col} number}
is marked via full reduplication of the noun, to convey the referent in a more broad, general sense.
Compare \cref{ex:i_have_pamats_chicken,ex:i_have_chickens}: the latter refers to \rom{carikcarik} (`chickens')
in a more broad sense, without mentioning any \textit{single} chicken.

\begin{example}
  \label{ex:i_have_pamats_chicken}
  \gloss
  curah & carry \\
  =ti & 1SG.NOM \\
  Pamat & Pamat \\
  ta- & ACC \\
  me= & 3SG.GEN \\
  carik & chicken \\
  \lit I carry Pamat's chicken(s).
  \tr I have Pamat's chicken(s).
\end{example}

\begin{example}
  \label{ex:i_have_chickens}
  \gloss
  curah & carry \\
  =ti & 1SG.NOM \\
  ta- & ACC \\
  \highlight{carik}^ & COL \\
  \highlight{carik} & chicken \\
  \tr I have chickens (in a general sense).
\end{example}

This reduplication also serves to turn any pronoun or demonstrative into a collective,
conveying a sense of `\textit{all} of X'. This is demonstrated in \cref{ex:he_will_kill_them,ex:all_sea}.
For pronominals, the pronoun clitic on verbs is still required although the pronoun subject might be reduplicated;
for demonstrative determiners, the referent noun need not be reduplicated too.

\begin{example}
  \label{ex:he_will_kill_them}
  \gloss
  mera & 3SG \\
  ta- & ACC \\
  \highlight{halwat}^ & COL \\
  \highlight{halwat} & 3PL \\
  ʔerat & FUT \\
  wahas & kill \\
  =∅ & 3SG.NOM \\
  \tr He will kill \textit{all} of them.
\end{example}

\begin{example}
  \label{ex:all_sea}
  \gloss
  parayaw & sea \\
  \highlight{ʔamuy}^ & COL \\
  \highlight{ʔamuy} & DEM.PROX \\
  puŋ- & PST \\
  santuk & dirty \\
  =wat & 3PL.NOM \\
  \tr \textit{All} of this ocean was dirty.
\end{example}
